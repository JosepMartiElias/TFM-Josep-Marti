%%%%%%%%%%%%%%%%%%%%%%%%%%%%%%%%%%%%%%%%%
% Masters/Doctoral Thesis 
% LaTeX Template
% Version 1.43 (17/5/14)
%
% This template has been downloaded from:
% http://www.LaTeXTemplates.com
%
% Original authors:
% Steven Gunn 
% http://users.ecs.soton.ac.uk/srg/softwaretools/document/templates/
% and
% Sunil Patel
% http://www.sunilpatel.co.uk/thesis-template/
%
% License:
% CC BY-NC-SA 3.0 (http://creativecommons.org/licenses/by-nc-sa/3.0/)
%
% Note:
% Make sure to edit document variables in the Thesis.cls file
%
%%%%%%%%%%%%%%%%%%%%%%%%%%%%%%%%%%%%%%%%%

%----------------------------------------------------------------------------------------
%	PACKAGES AND OTHER DOCUMENT CONFIGURATIONS
%----------------------------------------------------------------------------------------

\documentclass[11pt, twoside,a4paper]{Thesis} % The default font size and one-sided printing (no margin offsets)
%\makeatletter
%\let\ps@plain\ps@fancy
%\makeatother
\usepackage{placeins}
\usepackage{eurosym}
\usepackage[catalan]{babel}
\usepackage{float}
\usepackage{color}
\usepackage{mwe}
\usepackage{listings}    
\usepackage{etoolbox}  
\usepackage{pythonhighlight}
\usepackage{wrapfig}
\usepackage{indentfirst}
\usepackage{pgfgantt}
\usepackage{eurosym}
\usepackage{longtable}
\graphicspath{{Pictures/}} % Specifies the directory where pictures are stored
\makeatletter
\renewcommand{\@chapapp}{Capítol}
\renewcommand{\figurename}{Figura}
\renewcommand{\contentsname}{Índex}
\renewcommand{\listfigurename}{Llista de figures}
\renewcommand{\listfigurename}{}

\usepackage[square, numbers, comma, sort&compress]{natbib} % Use the natbib reference package - read up on this to edit the reference style; if you want text (e.g. Smith et al., 2012) for the in-text references (instead of numbers), remove 'numbers' 
\hypersetup{urlcolor=black, linkcolor=black} % Colors hyperlinks in blue - change to black if annoying
\title{\ttitle} % Defines the thesis title - don't touch this

\begin{document}

\frontmatter % Use roman page numbering style (i, ii, iii, iv...) for the pre-content pages

\setstretch{1.3} % Line spacing of 1.3

% Define the page headers using the FancyHdr package and set up for one-sided printing
%\fancyhead[LE,RO]{\thepage} % Clears all page headers and footers


%\rhead[]{\thepage} % Sets the right side header to show the page number
%\lhead[\thepage]{} % Clears the left side page header

\pagestyle{fancy} % Finally, use the "fancy" page style to implement the FancyHdr headers

\newcommand{\HRule}{\rule{\linewidth}{0.5mm}} % New command to make the lines in the title page

% PDF meta-data
\hypersetup{pdftitle={\ttitle}}
\hypersetup{pdfsubject=\subjectname}
\hypersetup{pdfauthor=\authornames}
\hypersetup{pdfkeywords=\keywordnames}





%----------------------------------------------------------------------------------------
%	TITLE PAGE
%----------------------------------------------------------------------------------------

\begin{titlepage}
	\begin{center}
		
		
		\textsc{\Large Treball final de Màster}\\[0.5cm] % Thesis type
		\textsc{\Large Master Universitari en Enginyeria Industrial}\\[0.5cm]
		\HRule \\[0.4cm] % Horizontal line
		{\huge \bfseries \ttitle}\\[0.4cm] % Thesis title
		\HRule \\[1.5cm] % Horizontal line
		
		\begin{minipage}{0.4\textwidth}
			\begin{flushleft} \large
				\emph{Autor:}\\
				{\authornames} % Author name - remove the \href bracket to remove the link
			\end{flushleft}
		\end{minipage}
		\begin{minipage}{0.4\textwidth}
			\begin{flushright} \large
				\emph{Tutor:} \\
				{\supname} % Supervisor name - remove the \href bracket to remove the link  
			\end{flushright}
		\end{minipage}\\[3cm]
		
		
		
		\groupname\\\deptname\\[2cm] % Research group name and department name
		
		{\large \today}\\[4cm] % Date
		%\includegraphics{Logo} % University/department logo - uncomment to place it
		
		\vfill
	\end{center}
	
\end{titlepage}

%----------------------------------------------------------------------------------------
%	DECLARATION PAGE
%	Your institution may give you a different text to place here
%----------------------------------------------------------------------------------------



%----------------------------------------------------------------------------------------
%	QUOTATION PAGE
%----------------------------------------------------------------------------------------


%----------------------------------------------------------------------------------------
%	ABSTRACT PAGE
%----------------------------------------------------------------------------------------

\addtotoc{Resum (inicial)} % Add the "Abstract" page entry to the Contents
\onehalfspace
\abstract{\addtocontents{toc}{\vspace{1em}} % Add a gap in the Contents, for aesthetics

La idea principal d'aquest projecte és dissenyar, programar i construir un robot capaç de dibuixar sobre qualsevol paper, de qualsevol dimensió, la figura o traçada que l'usuari desitgi. 
\\*
\\*
Per fer-ho, es vol dissenyar un robot de dues rodes motrius controlades per dos motors pas a pas, una roda boja per fer estable l'estructura, un suport per un punter (ja sigui un llapis, un retolador o qualsevol objecte que serveixi per dibuixar) situat al centre entre les dues rodes, una estructura de plàstic impresa amb tècniques d'impressió 3D i un microcontrolador \emph{Arduino Uno} per tal de controlar tot el sistema i realitzar les comunicacions amb l'ordinador de l'usuari. Tota la programació serà realitzada a través de la plataforma de software lliure d'\emph{Arduino}.
\\*
\\*
El primer objectiu serà trobar la manera de controlar els dos motors simultàniament per tal que segueixin la trajectòria desitjada. És molt important que aquest control sigui molt acurat ja que inicialment la posició del robot estarà controlada en llaç obert i per tant no es podran corregir el errors comesos i s'acumularan creant distorsió entre la trajectòria desitjada i la descrita pel propi robot. És per aquest motiu que s'utilitzaran motors pas a pas ja que en tot moment es pot saber la seva posició que vindrà donada pels passos produïts. 
\\*
\\*
Paral·lelament s'ha de dissenyar l'estructura del robot per així estudiar el seu model cinemàtic i conèixer el seu moviment per tal de traduir-ho a impulsos del motor i així controlar el posicionament.
\\*
\\*
Un cop conegut el funcionament del robot i el seu moviment cal aconseguir dissenyar un lector de trajectòries que pugui comunicar-se amb el robot per tal que aquest realitzi el moviment desitjat sobre el paper. La idea inicial és aconseguir traduir la trajectòria desitjada a \emph{G-code} amb el software lliure de dibuix i edició d'imatges \emph{Inkscape}. A partir d'aquest codi, s'ha de decidir quina serà la manera adequada per comunicar l'ordinador amb l'\emph{Arduino} per transmetre cada una de les ordres del codi i realitzar-les. Hi ha diferents opcions com poden ser mòduls \emph{wifi} o \emph{Bluetooth} o fins i tot a partir de memòries externes del tipus SD. 
\\*
\\*
Més endavant i si el temps ho permet, es podrà dissenyar alguna manera per tancar el llaç de control de posició i així corregir errors que es puguin produir i fer el sistema més precís.

}

\clearpage % Start a new page

%----------------------------------------------------------------------------------------
%	ACKNOWLEDGEMENTS
%----------------------------------------------------------------------------------------

\setstretch{1.3} % Reset the line-spacing to 1.3 for body text (if it has changed)

\acknowledgements{\addtocontents{toc}{\vspace{1em}} % Add a gap in the Contents, for aesthetics


}
\clearpage % Start a new page

%----------------------------------------------------------------------------------------
%	LIST OF CONTENTS/FIGURES/TABLES PAGES
%----------------------------------------------------------------------------------------

\pagestyle{fancy} % The page style headers have been "empty" all this time, now use the "fancy" headers as defined before to bring them back

\rhead[\emph{Índex}]{\thepage}
\lhead[\thepage]{\emph{Índex}} % Set the left side page header to "Contents"
\tableofcontents % Write out the Table of Contents

\lhead{\emph{Índex de figures}} % Set the left side page header to "List of Figures"
\listoffigures % Write out the List of Figures

\lhead{\emph{List of Tables}} % Set the left side page header to "List of Tables"
\listoftables % Write out the List of Tables

%----------------------------------------------------------------------------------------
%	ABBREVIATIONS
%----------------------------------------------------------------------------------------

%----------------------------------------------------------------------------------------
%	PHYSICAL CONSTANTS/OTHER DEFINITIONS
%----------------------------------------------------------------------------------------



%----------------------------------------------------------------------------------------
%	SYMBOLS
%----------------------------------------------------------------------------------------

\clearpage % Start a new page

\lhead{\emph{Symbols}} % Set the left side page header to "Symbols"

\listofnomenclature{ll} % Include a list of Symbols (a three column table)
{
mm & mil·limetres \\


% Symbol & Name & Unit \\
}

%----------------------------------------------------------------------------------------
%	DEDICATION
%----------------------------------------------------------------------------------------



%----------------------------------------------------------------------------------------
%	THESIS CONTENT - CHAPTERS
%----------------------------------------------------------------------------------------

\mainmatter % Begin numeric (1,2,3...) page numbering

\pagestyle{fancy} % Return the page headers back to the "fancy" style

% Include the chapters of the thesis as separate files from the Chapters folder
% Uncomment the lines as you write the chapters

% Chapter 1
\fancyfoot[LE,RO]{
	\includegraphics[scale=0.3]{logo1}
     }
\setlength\topmargin{8mm}
\onehalfspacing
\chapter{Introducció} % Main chapter title

\label{Chapter1} % For referencing the chapter elsewhere, use \ref{Chapter1} 

\rhead[\emph{Disseny, programació i implementació d'un robot de dibuix amb Arduino}]{\thepage}
\lhead[\thepage]{\emph{Disseny, programació i implementació d'un robot de dibuix amb Arduino}}

%----------------------------------------------------------------------------------------

Imaginem un arquitecte que pretén delinear sobre un terreny per tal d'indicar els espais que ocuparà la construcció. Aquest necessitarà contractar un topògraf per tal de realitzar-ho, fet que provoca un augment de pressupost i dels temps d'execució de l'obra. Suposem ara que es vol realitzar una instalació elèctrica en un carrer i, a l'hora de foradar el paviment, existeix el perill de fer malbé les canonades que hi passen per sota. Actualment, es disposa del plànol de canonades però és difícil determinar amb exactitud per on passen. Per altra banda, imaginem que es vol tallar sobre una planxa de fusta, de cartró, un taulell DM o algún altre material, un disseny específic. Per fer-ho primer cal traspassar el disseny a la superficie per tal de poder-ho tallar després. Per acabar, suposem que hi ha la necessitat de traçar les línies de qualsevol espai esportiu, com un camp de futbol o una pista d'atletisme, i es vol evitar que puguin quedar les línies desviades. Aquests són alguns exemples de situacions reals en les que es requereix un trasspas d'un disseny tècnic o plànol a una superficie plana. 

Aquest projecte vol estudiar la viabilitat de realitzar un dispositiu de baix cost, fàcilment transportable i que es pugui utilitzar per resoldre qualsevol d'aquestes situacions. Aixó donarà peu a una eina de dibuix que eliminarà les possibles limitacións de tamany que presenten les actuals. S'estudiarà una primera aproximació a la solució, realitzant un prototip capaç de convertir arxius SVG (Scalar Vector Graphics) que contiguin dibuixos tècnics, al movient d'un robot capaç de dibuixar sobre una superficie plana sense necessitat d'utilitzar grans estructures. 

S'intentarà realitzar a partir de la teconologia de la qual es disposa: en primer lloc cal estudiar la manera de convertir un disseny tècnic a una trajectòria. A continuació, s'ha de programar el robot per aconseguir que segueixi aquesta trajectòria. Per fer-ho, primer cal escollir quins components poden aconseguir aquest moviment de la manera més precisa possible i estudiar el seu funcionament, així com definir un mètode de control que els pugui fer funcionar conjuntament. Per altra banda, és necessari realitzar el disseny de l'estructura que incorpori tots aquests elements per tal de crear un robot móbil de petites dimensions. Un cop realitzat el disseny i construit el robot, s'ha d'estudiar la precisió del mateix per veure si compleix amb les especificacions.

\section{Objectius}

Aquest projecte pretén dissenyar, programar i construir un robot capaç de dibuixar sobre un paper, de qualsevol dimensió, la figura o traçada que l'usuari desitgi. Per tal d'aconseguir-ho, s'han un definit uns objectius principals:

\begin{itemize}
	\item Avaluar la viabilitat de realitzar el robot com un sistema de llaç obert.
	\item Dissenyar i construir un robot des de zero.
	\item Dotar el robot dels mitjants de software necessaris per a la realització de la seva funció. 	
\end{itemize}

Per poder complir els objectius principals s'han d'assolir primer alguns objectius secundaris. 

\begin{itemize}
	\item Extreure el GCode associat a la trajectòria de la figura que l'usuari vol dibuixar.
	
	\item Crear l'algoritme encarregat de traduir un arxiu de text que conté el GCode associat a una trajectòria, a les ordres necessàries per moure el robot.
	
	\item Seleccionar els elements que millor s'ajustin a les especificacions del projecte, per tal de garatir el funcionament i augmentar la precisió.
	
	\item Aprofundir en la programació tant d'Arduino com de Python, per poder realitzar el projecte, i en el disseny CAD amb Solidworks per tal de dissenyar les peces que s'imprimiran en 3D. Per altra banda, cal aprendre a utilitzar altres programes o llenguatges, com l'Inkscape, per a la realització de dibuixos, i el Latex per a la realització de la memòria.
	
	\item Establir un protocol de comunicació entre el robot i l'ordinador.
	
	\item Programar una aplicació que permeti a l'usuari treballar amb el robot de manera més senzilla i entenedora.
\end{itemize} 



\section{Abast i resultat esperat}

L'abast d'aquest projecte és dissenyar, programar i construir un prototip del robot que funcioni en llaç obert, capaç de seguir les trajectòries desitjades per l'usuari, i programar l'aplicació per utilitzar-lo. A partir d'aquí, cal aclarir alguns punts que queden fora d'aquest abast per motius diferents:

\begin{itemize}
	\item En primer lloc, cal deixar clar que es vol estudiar la viabilitat de realitzar els objectius amb un sistema en llaç obert i, per tant, queda fora de l'abast del projecte l'estudi d'implementació d'un sistema que permeti tancar el llaç de control. Aixó s'ha decidit així per centrar l'estudi en la part inicial de la construcció i programació del robot, ja que el volum de feina que pot comportar tancar el llaç de control podria constituir un altre treball per si mateix, i per tant s'ha descartat. 
	
	\item Aquest projecte vol trobar una primera solució i estudiar-ne la seva viabilitat, per tant queda fora de l'abast aconseguir millorar les seves prestacions, ja que aixó voldira dir realitzar grans inversions en millores de materials i components, i una feina de redisseny per aconseguir-ho.
	
	\item S'ha descarat la creació d'un mètode d'extracció del GCode associat a una imatge, ja que ja existeixen eines que ho realitzen, com el programa Inkscape. Gràcies a l'existència d'aquests s'ha decidit no crear un nou software per aconseguir-ho i així destinar aquest temps a altres tasques més importants.
	
	\item Al ser un treball d'estudi i en el qual es vol crear un primer prototip per estudiar-ne la viabilitat s'ha descartat l'opció de crear una placa de circuit imprés (PCB) que integri tots els components que el formen, ja que és un procés de millora continua i s'han d'afegir nous components al llarg del temps.   
\end{itemize}












% Chapter 2
\setlength\topmargin{8mm}
\onehalfspacing
\chapter{Principis de funcionament} % Main chapter title

\label{Chapter2} % For referencing the chapter elsewhere, use \ref{Chapter1} 

\rhead[\emph{Disseny, programació i implementació d'un robot de dibuix amb Arduino}]{\thepage}
\lhead[\thepage]{\emph{Disseny, programació i implementació d'un robot de dibuix amb Arduino}}

%----------------------------------------------------------------------------------------



%----------------------------------------------------------------------------------------

Aquest capítol és una introducció al funcionament del robot. S'explicaran els trets principals per tal d'entendre quin és el procés que s'ha de seguir per tal d'aconseguir dibuixar. Més endavant s'explicarà com s'han realitzat totes les etapes del projecte i per què s'ha fet d'aquesta manera. 

L'objectiu d'aquest dispositiu és aconseguir traspassar un disseny tècnic o dibuix realitzat per ordinador a la vida real sobre un paper. L'avantatge que presenta és que, al ser mòbil, a priori no té limitacions d'espai, pot fer-ho a l'escala que es desitgi mentre sigui sobre una superifície plana. El primer pas serà definir quins elements consituiran el robot i faràn possible aquest moviment. 

El robot presenta dues rodes motrius accionades per dos motors pas a pas que s'encarreguen de realitzar un moviment controlat i precís. Aquests estaràn controlats per un driver que actua com a etapa de potència i facilitarà el seu control, ja que funciona a partir de només 2 pins digitals, l'un controlador de l'step i l'altre de la direcció. Per tal de fer possible el moviment vertical del retolador s'utilitzarà un microservo que accionarà un mecansime per convertir la seva rotació en el moviment vertical del retolador dins una guia. Tot el conjunt estarà controlat per un microcontrolador Arduino UNO que es comunicarà amb l'ordinador mitjaçant una connexió Bluetooth. Tot el circuit s'alimentarà amb una bateria LiPo de 10000 mha amb dues sortides de 5V i d'1 i 2 A respectivament. Aquesta conexió es representa al següent diagrama:

\begin{figure}[H]
	\centering
	\includegraphics[scale=0.4]{RobotFritz}
	\caption{Diagrama de connexió dels components realitzada amb el programa Fritzing \cite{FritzingBib}.}
	\label{fig:connexio1}
\end{figure}

Un cop escollits els elements que formaran el robot s'haurà de dissenyar una estructura que incorpori tots aquests i permeti al robot fer la seva funció, intentant sempre optimitzar el tamany per fer-lo el més petit possible. Caldrà dissenyar diferents peces: un xassís, les rodes, una roda davantera i un mecanisme capaç de realitzar el moviment vertical del retolador. Aixó es realitzarà a partir de SolidWorks i s'imprimirà en 3D a l'aula Rep Rap de l'escola. 

\begin{figure}[H]
	\centering
	\includegraphics[scale=0.1]{RobotFoto}
	\caption{Fotografia del robot.}
	\label{fig:foto}
\end{figure}

Per últim caldrà crear el software controlador del robot. Des de l'ordinador, es processarà la imatge i s'extreuàa el GCode de la trajectòria, d'aquest GCode s'extreuran les ordres de moviment i es traduiran a passos del motor, que s'enviaran a l'Arduino per una connexió Bluetooth i aquest s'encarregarà de moure els motors. 

\begin{figure}[H]
	\centering
	\includegraphics{Flux}
	\caption{Diagrama de flux del funcionament general.}
	\label{fig:flux}
\end{figure}

Aquest és el funcionament bàsic del robot i, a continuació, s'explicarà cadascuna de les parts de manera més detallada. 













\input{Chapters/Chapter3}
% Chapter 4
\setlength\topmargin{8mm}
\onehalfspacing
\chapter{Disseny de l'estructura} % Main chapter title

\label{Chapter4} % For referencing the chapter elsewhere, use \ref{Chapter1} 

\rhead[\emph{Disseny, programació i implementació d'un robot de dibuix amb Arduino}]{\thepage}
\lhead[\thepage]{\emph{Disseny, programació i implementació d'un robot de dibuix amb Arduino}}

En aquest apartat es descriu el disseny del robot, la creació de l’estructura i la distribució dels components, explicant el perquè d’aquesta distribució i les premisses que s’han tingut en compte a l'hora de dissenyar les diferents peces en Solid Works per tal d'imprimir-les en 3D posteriorment a l'aula Rep Rap de la universitat.   

\begin{figure}[H]
	\centering
	\includegraphics[scale=0.6]{RobotSW.png}
	\caption{Assemblatge del robot en SolidWorks (exceptuant la bateria, el cablejat i el módul Bluetooth).}
	\label{fig:RobotSW}
\end{figure}

\section{Xassís}

El xassís és el cos del robot, incorpora totes les altres peces i el seu disseny és una de les parts fonamentals del treball. És l’encarregat de centrar els eixos dels motors i fixar la seva posició, d’incorporar la bateria, el microcontrolador, els drivers i tot el circuit elèctric, i d'assegurar la seva posició fixa durant el moviment del robot. A continuació es mostra una imatge del disseny del xassís on s'indiquen les parts més importants que després s'explicaran. 

\begin{figure}[H]
	\centering
	\includegraphics{xassis}
	\caption{Parts del xassís.}
	\label{fig:xassis}
\end{figure}

L’objectiu era dissenyar un xassís amb les dimensions més petites possibles, i aquestes van venir donades directament per la mida dels dos motors. S’ha dissenyat  amb una amplada de 104 mm tenint en compte que ha de suportar els dos motors que tenen una llargada de 34 mm i que a l’espai entre els dos cal adaptar el suport del retolador pel qual s'han deixat 28 mm. Els altres 8 mm restants estan reservats per les fixacions dels motors. Per tal de no posar unes limitacions molt restrictives, s’ha decidit realitzar un forat de 20 mm de diàmetre (detall 1 de la imatge \ref{fig:xassis}) per l’element de dibuix, per  adaptar el suport a les mides del retolador o estri escollit. És fonamental que aquest forat estigui centrat entre els dos motors i amb un eix perpendicular i coincident amb els eixos dels motors que han d’estar alineats. D’aquesta manera s’aconsegueix l’amplada mínima, ja que la resta d’elements són de dimensions més reduïdes i ,per tant, s’han posicionat de manera centrada per mantenir centrat el centre de masses. 

Per tal de fixar els motors s’han incorporat uns suports verticals (detall 2) que fixen amb cargols la cara frontal del motor (on neix l’eix) evitant així que pugui desplaçar-se i mantenint d’aquesta manera els dos motors sempre alineats. Aquests suportss es van imprimir incorporats directament al xassís, per minimitzar els possibles errors de muntatge o descentratge. De la mateixa manera, aprofitant aquests suports i per optimitzar l’espai, s’ha incorporat una petita part plana horitzontal (detall 3) sobre els motors per instal·lar-hi els drivers dels mateixos i així mantenir-ho junt i facilitant el reconeixement del driver corresponent a cada motor. Cal destacar que aquests suports no tanquen la part superior de tal manera que es poden muntar i desmuntar els motors de manera més còmoda sense, per exemple, desmuntar el suport del retolador. 

La longitud del xassís ha de ser la suficient per col·locar-hi els motors, el servomotor encarregat d’aixecar el retolador, la bateria, el microcontrolador Arduino i la roda davantera. D’aquesta manera, el xassís final té una llargada de 108,68 mm i la següent distribució: 

En primer lloc es col·loquen els motors de 35 mm d'amplada. Seguidament, hi ha un espai reservat al suport fixe del bolígraf que s’ha dissenyat com una peça diferent per tal de poder adaptar-la a altres utillatges sense haver de modificar el xassís per complet. Disposa d’una base per enganxar el servomotor fixant la seva alçada de manera que l’eix del servo es mantingui a la mateixa alçada que l’extrem del suport fixe, per tal d’exercir la força per aixecar el retolador sobre el suport mòbil des de la posició horitzontal per gaudir del recorregut màxim del moviment si fos necessari. 

En següent lloc, s’ha dissenyat un espai reservat per la bateria (detall 4) que compte amb un petit compartiment que manté la bateria en posició vertical i fixa per evitar que caigui o pugui interrompre el funcionament del robot. Aquest compartiment té una mida de 25 x 65 mm amb tres parets de 20 mm d’alçada i la paret posterior de 90 mm que s’aprofita per fer de base de l’Arduino (detall 5). Es pot observar que les mides són uns mil·límetres més grans que la pròpia bateria, ja que s’ha deixat l’espai suficient per tal de poder posar els cargols per fixar l’Arduino. És per aquesta raó que s’ha deixat un petit espai entre les parets laterals i la paret verical per poder manipular els cargols amb la bateria fixada si fos necessari. S’ha decidit instal·lar la bateria al punt mig del robot per distribuir la massa al llarg del xassís i evitar que el centre de masses quedi a la línia de l’utillatge de dibuix i el robot pugui bolcar en algun punt de la trajectòria degut a algun moviment concret.   

L’Arduino, com s’acaba d’explicar, es col·loca a la paret vertical del compartiment de la bateria de manera que queda fixat verticalment, optimitzant així els espais. S’han dissenyat els forats per tal de col·locar l’entrada del port serial a la part superior fent possible d’aquesta manera la programació de la placa un cop s’ha implementat tot el prototip, ja que sinó s'hauria de desmontar cada cop que es volgués programar. Al lateral de la placa s’han instal·lat dues barres de corrent de protoboard per tal de poder alimentar des de la mateixa sortida de la font un motor conjuntament amb la placa i l’altre motor amb el servomotor. 

Per acabar, a la part davantera s’ha deixat l’espai lliure (detall 6) suficient per instal·lar-hi una roda boja que actuarà com a punt de suport per estabilitzar el robot. S'ha arrodonit la forma davantera per motius completament estètics i s'ha afegit un petit sortint circular a la part posterior (on hi ha els motors) per tal de facilitar la fixació de la guia del retolador.

\section{Guia del retolador} \label{sec:guia}

Aquest suport està dissenyat per funcionar com a guia pel retolador, de manera que el seu desplaçament sigui vertical i centrat per minimitzar l’error de descentratge del mateix. Consta d’una part plana en forma de T i un tub vertical que farà de guia pel retolador. Va fixada al xassís amb 3 cargols. La guia està expressament dissenyat per treballar amb un retolador Edding 1200 (presentat a l’apartat \ref{sec:retolador}) i s’ha dissenyat de manera independent del xassís per així poder fer-la a mida de l’estri que es vulgui utilitzar sense haver de tornar a imprimir tot el cos del robot. 

\begin{figure}[H]
	\centering
	\includegraphics{guia}
	\caption{Imatge del disseny de la guia.}
	\label{fig:guia}
\end{figure}

També compta amb un petit suport per tal d’enganxar el servomotor de manera que l'alçada de l’eix d’aquest sigui lleugerament inferior a l’alçada de la part superior de la guia i aconseguir així que el moviment del servo pugui ser màxim, des de la posició horitzontal amb l’estri actiu fins a una posició mitja amb l’estri aixecat és a dir, sense actuar. No és més que un petit graó a la part dreta del suport de la mids del servo i d'alçada de 5 mm per aconseguir la posició desitjada del servo, que s’ha fixat directament amb cinta adhesiva de doble cara ja que el seu esforç no és massa elevat i queda fixat perfectament sense haver de foradar el suport imprès en 3D. 

\begin{figure}[H]
	\centering
	\includegraphics{GuiaServo}
	\caption{Servomotor muntat sobre la guia.}
	\label{fig:guiaservo}
\end{figure}


\section{Mecanisme d'elevació del retolador} \label{sec:suportmobil}

Aquest mecanisme consta d’una petita placa fixada a certa altura del retolador que actua com a superfície per tal de poder transformar el moviment circular del servo en el moviment vertical de l’utillatge. L’alçada a la qual s’ha de col·locar ve donada per la guia del retolador fixa al xassís (apartat \ref{sec:guia}), assegurant d'aquesta manera que quan el servomotor està en posició de dibuix aquest suport es recolzi a la cara plana superior de la guia i el retolador estigui en contacte amb el paper. 

S’ha dissenyat amb una part més allargada (detall 1 de la imatge \ref{fig:suport}) per tal d’assegurar el contacte i l’acció del servomotor al aixecar el braç, i a l'altre costat s’ha deixat un espai buit i dues parets verticals foradades (detall 2) per poder fixar l’estri amb l’ajuda d’un cargol un cop ajustada l'altura a la qual s'ha de fixar. Aquesta alçada s'ha d'ajustar mantenint el servo en posició activa, el retolador tocant al paper i el suport recolzat a la cara superior de la guia. L’eix vertical (detall 3) serveix per tal d’assegurar la perpendicularitat de l’estri amb la base del suport. 

\begin{figure}[H]
	\centering
	\includegraphics{suport}
	\caption{Imatge del disseny del suport pel mecanisme d'elevació.}
	\label{fig:suport}
\end{figure}

\section{Roda boja davantera}

Com a tercer punt de suport i per estabilitzar el robot, s’ha dissenyat una roda boja de manera que no intervingui en el moviment del robot, permetent el lliscament en qualsevol direcció. Consta de dues parts, el cos imprès en 3D i una bola de vidre que gira lliureament a l'interior. El cos assegura el contacte amb la bola en només 4 punts d’una circumferència paral·lela a la base fent així que el lliscament sigui més fàcil i minimitzant la força de fricció. També assegura que la bola es mantingui dins la cavitat quan s’aixeca el robot fet que provoca que la bola entri a pressió. La cara superior és una cara plana per facilitar la impressió 3D de la peça amb dos forats per fixar-la amb cargols al xassís i un forat per on es pot observar la bola per poder treure-la aplicant una petita pressió. 

\begin{figure}[H]
	\centering
	\includegraphics{bolaboja}
	\caption{Suport de la roda boja davantera.}
	\label{fig:bolaboja1}
\end{figure}

L’alçada de la roda, juntament amb les rodes motrius, són les encarregades de fixar l’alçada del robot, de manera que s’han dissenyat conjuntament per tal d’assegurar que la base sigui paral·lela al paper i mantenir l’eix del retolador vertical i centrat.

\begin{figure}[H]
	\centering
	\includegraphics{pla}
	\caption{Base del robot paral·lela al terra.}
	\label{fig:pla}
\end{figure}



\section{Rodes motrius}

Les rodes són una part fonamental del disseny, ja que són les encarregades del moviment del propi robot. Hi ha quatre aspectes bàsics per a la seva construcció: minimitzar la superfície de contacte amb el paper, assegurar la perpendicularitat i concentricitat amb l’eix, evitar el lliscament perquè giri de manera solidaria a l'eix i optimitzar el diàmetre per aconseguir un pas més precís disminuint al màxim el diàmetre. 

Per minimitzar la superfície de contacte amb el paper, s’ha decidit utilitzar una junta tòrica de goma d’1,5mm de diàmetre, que actua de la mateixa manera que els neumàtics d’un cotxe, evitant el lliscament amb el paper i, al ser circular, es minimitza el contacte fent-lo puntual o gairebé puntual. Així s’evita que un major contacte de la roda pugui actuar en contra del moviment del robot, creant forces de fricció que evitin la rotació del robot sobre el punt de contacte. Amb una superfície puntual també es millora la precisió dels moviments circulars, ja que es calculen a partir de la distància entre el retolador i la superfície de contacte i, per tant, com més petita sigui aquesta última, més exacta i constant serà aquesta distància durant el moviment radial de la roda. Per tal de mantenir el neumàtic a la roda, aquesta s’ha imprès directament amb un petit conducte amb la forma de mig neumàtic (detall 1 de la figura \ref{fig:roda}) aconseguint així que encaixin. 

En la pròpia impressió de la roda, s’ha afegit un eix transversal (detall 2) del mateix diàmetre que l’eix del motor per assegurar-ne d’aquesta manera la perpendicularitat i concentricitat. És un fet clau per assegurar el correcte funcionament, perquè, d’igual manera que al punt anterior, s’ha d’assegurar que la distancia entre les rodes sigui sempre la mateixa al llarg de tots els punts de la rotació de les mateixes, ja que és així com s’ha calculat la trajectòria. 

Un altre element del disseny són les dues petites parets verticals foradades (detall 3) que surten de l’eix i el forat que es pot observar al cos de la roda. Això permet que amb un cargol i una femella es pugui fixar la roda a l'eix, ja que es crea una deformació ínfima que redueix el forat a la mida exacte de l'eix del motor fent que quedin fixats sense alterar la forma de la roda. A  més, s'ha incorporat un forat a la pròpia roda (detall 4) que permet la manipulació dels cargols del motor sense la necessitat de treure la roda. 

Per acabar, el diàmetre de la roda s’ha intentat minimitzar de manera que es disminueix l’avanç de la roda per cada pas, augmentant així la precisió. Això s’ha realitzat tenint en compte la distància entre els eixos dels motors i la base, més una distància prudencial de 4,5 mm per poder col·locar els cargols necessaris per fixar els suports de l’utillatge de dibuix i la roda boja i mantenir-los separats del paper. D’aquesta manera s’han dissenyat unes rodes de 50,5 mm que sumat els neumàtics de goma acaba sent de 51,25 mm. 

\begin{figure}[H]
	\centering
	\includegraphics{roda}
	\caption{Disseny de la roda.}
	\label{fig:roda}
\end{figure}


\section{Cargols, femelles i tubs auxiliars}

Per l’assemblatge de tot el prototip s’han utilitzat cargols de mètrica M3 i les femelles adients per tal de fixar-los. S’han utilitzat un total de 14 cargols M3x10mm de llargada, 3 per fixar el suport del retolador, 3 per fixar la placa Arduino al xassís i 8 per fixar els motors als suports del xassís, 4 per cada motor. Per altre banda, s’han emprat 4 cargols M3x30mm per fixar els drivers (2 per cadascun) i 2 més per la roda boja davantera, que es podia fixar amb cargols més curts, però s’han utilitzat aquests per aprofitar els cargols disponibles. Tots els cargols porten una femella per tal de poder fer fixa la unió menys en el cas dels motors. En aquests, s'han utilitzat les femelles per escurçar la longitud del cargol i evitar el contacte amb els cargols posteriors del motor, ja que comparteixen eix. 

Per altra banda, s'han dissenyat i imprés en 3D uns tubs de plàstic, per tal d’elevar tant els drivers com la placa Arduino. En primer lloc, s’han aixecat els drivers 23 mm respecte els seus suports per permetre d’aquesta manera la correcta connexió dels cables, ja que aquesta connexió es realitza per la part inferior del mateix. Per altra banda, el microcontrolador s’ha separat 2 mm respecte la paret per així evitar el contacte dels pins soldats i evitar possibles sobreescalfaments. 

\begin{figure}[H]
	\centering
	\includegraphics{tubs}
	\caption{Tubs auxiliars i exemples de cargols M3 i femelles utilitzats.}
	\label{fig:tubs}
\end{figure}




  


\setlength\topmargin{8mm}
\onehalfspacing
\chapter{Planificació temporal i costos} % Main chapter title

\label{Chapter5} % For referencing the chapter elsewhere, use \ref{Chapter1} 

\rhead[\emph{Disseny, programació i implementació d'un robot de dibuix amb Arduino}]{\thepage}
\lhead[\thepage]{\emph{Disseny, programació i implementació d'un robot de dibuix amb Arduino}}

\section{Planificació temporal}

En aquest capítol es presenta la programació temporal que ha seguit el projecte. S'ha dividit en diferents etapes que s'han anat realitzant per assolir els objectius proposats i s'han representat en un diagrama de Gantt. Aquestes etapes són:

\begin{itemize}
	\item A: Documentació prèvia: Aquesta va ser una etapa de recerca en la que es van estudiar mecanismes amb un funcionament similar, es va fer la tria dels components que millors prestacions tenien i oferien millor precisió, i es va començar a definir el mètode amb el que funcionaria el robot, basant-se en GCode com una màquina CNC.
	
	\item B: Compra de components: S'han realitzat compres de components durant tot el procés de construcció i muntatge, però s'en remarquen dues etapes especialment que consisteixen, en primer lloc, en la compra dels motors i el drivers, que va permetre començar a dissenyar i programar el robot, i, més endavant, la compra del mòdul Bluetooth que va permetre començar a treballar en el protocol de comunicació entre l'ordinador i l'Arduino. 
	 
	\item C: Aprenentatge i proves de programació de l'Arduino: Durant aquestes setmanes, es va aprofundir en la programació de l'Arduino i les primeres proves. Es va aprendre el funcionament dels motors i com controlar-los.
	
	\item D: Estudi de la dinàmica del robot: En aquesta etapa es va estudiar la dinàmica del robot i es va definir quin seria el seu funcionament, com es controlarien els motors i com es realitzaria el moviment. 
	 
	\item E: Proves de disseny: Abans de realitzar el disseny definitiu, es van fer diferents proves i maquetes per tal d'assegurar el correcte funcionament del robot i definir els punts crítics del disseny. 
	
	\item F: Disseny de les diferents peces: Un cop realitzades les proves prèvies i definits els requeriments de l'estructura, es va realitzar el disseny dels elements bàsics de l'estructura. Aquest està dividit en 2 períodes: el primer pel disseny del xassís i les rodes i el segon  pel disseny dels altres components com la guia i el suport del retolador.
	
	\item G: Impressió 3D i muntatge: Es van imprimir aquests dissenys en 3D i es va realitzar el mutatge de les diferents peces que conformen el robot. 
	
	\item H: Programació definitva: Gràcies als coneixements adquirits a l'etapa C i als coneixements previs de programació amb Python, es va dissenyar tot el software controlador del robot. Cal remarcar que es van anar aprenent nous conceptes de programació al llarg de tot el projecte, no només a l'etapa C.
	
	\item I: Creació de l'aplicació: Amb la programació del robot feta, es va decidir implementar una aplicació per millorar l'experiència de l'usuari. Durant aquestes 3 setmanes es va aprendre a utilitzar les eines adients i es va crear la GUI. 
	
	\item J: Redacció de la memòria: Ja a les primeres setmanes del projecte es va começar a redactar, però no va ser fins a les etapes finals que no es va començar a escriure el cos del treball fet, ja que primer calia crear els programes adients. 
\end{itemize}

\begin{figure}[H] 
	\centering
	\begin{ganttchart}[vgrid,hgrid, bar/.append style={fill=gray!30}, x unit=0.7cm, y unit chart=0.8cm]{1}{20},
		
		\gantttitle{2017}{20} \\
		\gantttitle{Febrer}{4}
		\gantttitle{Març}{4}
		\gantttitle{Abril}{4}
		\gantttitle{Maig}{4}
		\gantttitle{Juny}{4} \\
		\ganttbar{A}{1}{4} \\
		\ganttbar{B}{5}{6}
		\ganttbar[bar/.append style={fill=gray!15, dashed,} ]{}{7}{12}
		\ganttbar{}{13}{14}  \\
		\ganttbar{C}{5}{11} \\
		\ganttbar{D}{6}{8} \\ 
		\ganttbar{E}{6}{9} \\ 
		\ganttbar{F}{8}{10} \\
		\ganttbar{G}{11}{11}
		\ganttbar[bar/.append style={fill=gray!15, dashed}    ]{}{12}{12}
		\ganttbar{}{13}{14}\\
		\ganttbar{H}{11}{17} \\
		\ganttbar{I}{15}{17} \\ 
		\ganttbar{J}{13}{19}
		\ganttbar[bar/.append style={fill=gray!15, dashed}    ]{}{3}{12}
	\end{ganttchart}
	\caption{Diagrama de Gantt de la programació temporal del projecte.}
	\label{fig:gantt}
\end{figure}
\setlength\topmargin{8mm}
\onehalfspacing
\chapter{Resultats} % Main chapter title

\label{Chapter6} % For referencing the chapter elsewhere, use \ref{Chapter1} 

\rhead[\emph{Disseny, programació i implementació d'un robot de dibuix amb Arduino}]{\thepage}
\lhead[\thepage]{\emph{Disseny, programació i implementació d'un robot de dibuix amb Arduino}}

En aquests capítol es presentaran els diferents resultats que s'han obtingut a partir de les diferents proves que s'han realitzat amb el robot. També s'explicaran els errors que es poden observar i a què es deuen. Els errors més comuns són errors de calibratge, ja que aquest pot variar entre un ús i un altre. Les úniques variables de les quals depèn el moviment i que, per tant, cal calibrar són el diàmetre de les rodes i la distància de les rodes al punter del retolador. 

Cal destacar un error present en tots els dibuixos, i és la oscil·lació de la recta. Aquest traç tremolós apareix degut a la vibració dels motors que fan que tota l'estructura vibri. A part d'això, tot i ser mínim, existeix un cert joc entre la guia i el retolador per permetre que llisqui sense resistència, i això permet un moviment no desitjat de la punta del retolador que pot crear traçades lleugerament oscil·lants.

\begin{itemize}
	\item Per començar, s'ha realitzat una línia recta de 100 mm seguint la comanda G01 X0 Y100. Amb aquesta primera prova es pretén verificar el calibratge del robot. Al ser només una línia recta, l'única variable que hi intervé és el diàmetre de la roda segons les eqüacions \ref{eq:dist}, \ref{eq:pas}, \ref{eq:steps} exposades a l'apartat \ref{sec:G00}.
	\begin{figure}[H]
		\centering
		\includegraphics{resultatLinia}
		\caption{Dibuix d'una línia recta de 100 mm.}
		\label{fig:linia}
	\end{figure}
	Com es pot observar a la imatge s'ha realitzat la prova dues vegades. En primer, lloc s'ha relitzat la línia verda (a) i s'ha comprovat que la mida és de 102 mm enlloc dels 100 esperats. Això és degut a la mala calibració de l'aparell, i s'ha tornat a calibrar amb l'ajuda d'un peu de rei. Amb el nou diàmetre s'ha pogut realitzar una segona línia, la blava (b), que, aquest cop sí, mesura 100 mm. 
	
	\item La següent prova consisteix en la realització d'un quadrat de 100 mm de costat. Aquest cop s'ha realitzat amb el robot ja calibrat, tant el diàmetre de les rodes com la distància entre aquestes i el centre. El resultat és el següent:
	\begin{figure}[H]
		\centering
		\includegraphics{resultatQuadrat}
		\caption{Dibuix d'un quadrat de 100 mm de costat.}
		\label{fig:quadrat}
	\end{figure}
	Es pot observar que el resultat és bo, tot i que no acaba de tancar la silueta del tot com s'observa a la cantonada inferior esquerra. Es dedueix que aquest error que es comet als vèrtexs, ja que tots els costat mesuren 100 mm. Això és degut a l'acumulació de l'error que es comet. No és un sistema perfecte, i els passos no són infinitessimals, per la qual cosa sempre es comet un petit error. Aquest error és molt més visible als angles que no pas a les rectes, ja que a les línies rectes només afecta a les distàncies, mentre que l'altre distorssiona tota la imatge. 
	
	\item A partir d'aquí ja s'ha treballat amb figures més complicades creades amb Inkscape. Primer s'ha realitzat una estrella i s'ha obtingut el següent resultat:
	\begin{figure}[H]
		\centering
		\includegraphics[scale=0.9]{resultatEstrella}
		\caption{Dibuix d'una estrella (Dibuix en Inkscape (a), Dibuix del robot (b)).}
		\label{fig:estrella}
	\end{figure}
	Com es pot veure, la respresentació és bastant acurada, tot i que es torna a observar l'error en l'angle de gir. En aquest cas, però, al girar en sentit invers a cada vèrtex es compensa en certa mesura aquest error i la distorsió de la imatge és baixa. Es pot veure que tampoc s'aconsegueix tancar el contorn de la figura. 
	
	\item Per comprovar el funcionament de les comandes G02 i G03 s'ha realitzat una figura simple amb l'ajuda de l'aplicació creada. Aquesta figura es basa en una línia recta, mitja circumferència realitzada amb un moviment G02 en sentit horari, mitja circumferència realitzada amb una comanda G03 en sentit antihorari i dues rectes més que tanquen el contorn. El resultat és el segûent:
	\begin{figure}[H]
		\centering
		\includegraphics{resultatFuncionara}
		\caption{Dibuix d'una figura amb arcs de circumferència.}
		\label{fig:funcionara}
	\end{figure}	
	L'error en l'angle es manté aproximadament igual que a les figures anteriors. Aquí, però, es pot observar un moviment horitzontal no desitjat de la punta del retolador que es presenta al final del segon arc de circumferència. Això pot ser causat pel joc del retolador, i pel mal centratge de la rotació.
	
	\item Per acabar, s'ha representat una lletra \textit{i} per tal de provar figures més complicades i moviments de posicionament. Després de crear-la a l'Inkscape s'ha aconseguit aquest resultat:
	\begin{figure}[H]
		\centering
		\includegraphics{resultatLletra}
		\caption{Dibuix d'una lletra \textit{i} en Inkscape (a) i dibuixada pel robot (b).}
		\label{fig:Lletra}
	\end{figure}
	La forma de la lletra es manté bastant similar a l'original, es pot veure com s'ha pogut tancar el cos i que manté una forma molt aproximada. Cal destacar, però, que hi ha diferents errors. En primer lloc, cal dir que la línia diagonal que no pertany al dibuix s'ha creat degut a un error de muntatge, ja que a l'acabar s'ha desenganxat l'accionador del servo que s'encarrega d'aixecar el retolador. Aquest error s'ha solucionat enganxant aquest accionador amb cola de contacte. Aquí el principal error s'acumula a la realització del punt de la \textit{i}. Al crear la trajectòria en Inkscape, aquest no detecta el punt com una circumferència perfecta i la crea a partir de molts arcs de petites dimensions. Això provoca constants aturades del robot per tal de realitzar rotacions molt petites i l'error que s'acumula és molt gran. Al ser la primera part que es dibuixa, distorsiona tota la imatge que queda desviada. Això s'observa al punt, veient que no s'acaba de tancar la silueta per la part superior. A l'arribar a l'últim punt de la circumferència que dibuixa el robot, aquest gira 90 graus, però com que aquest punt no coincideix amb l'inicial, la direcció del robot no és l'esperada. Això provoca aquesta desviació. 
	
	Per comprovar-ho, s'ha modificat el GCode i s'ha definit el punt com una cirumferència perfecta creada a partir de dos arcs de mitja cirumferència. El resultat és el següent:
	\begin{figure}[H]
		\centering
		\includegraphics{resultatLletra2}
		\caption{Dibuix de la lletra \textit{i} millorat.}
		\label{fig:Lletra2}
	\end{figure}
	El canvi és clar: l'error més elevat es comet al realitzar els petits moviments circulars del punt. Així s'aconsegueix un punt més definit i una lletra més ben posicionada.
	
	Els errors que es poden observar es localitzen principalment en el calibratge del robot. A part dels aspectes que s'han esmentat, també existeixen possibles errors de disseny i construcció que poden provocar aquests errors, per exemple un mal centratge de les rodes, que aquestes no siguin perfectament perpendiculars a l'eix, que existeixi un petit lliscament entre l'eix i la roda o la deformació de les juntes, que actuen com a neumàtic, que pot ser diferent al llarg del recorregut. També és possible que a causa de la fricció algún pas del motor no es realitzi com toca i això provoqui aquest error.  
	
	Per millorar els errors de disseny o vibracions del motor caldria redissenyar les peces, adquirir nous components amb millors prestacions i millorar els materials que el composen, per exemple. Això reduiria l'error en llaç obert considerablement. Per altra banda, es podrien estudiar les diferents vies que existeixen per tancar el llaç de control i intentar corregir aquest error com s'explica més endevant en l'apartat \ref{sec:treballsfuturs} de possibles treballs futurs, ja que no entra a l'abast d'aquest projecte. 
	
	
	
\end{itemize}




\setlength\topmargin{8mm}
\onehalfspacing
\chapter{Conclusions i treballs futurs} % Main chapter title

\label{Chapter7} % For referencing the chapter elsewhere, use \ref{Chapter1} 

\rhead[\emph{Disseny, programació i implementació d'un robot de dibuix amb Arduino}]{\thepage}
\lhead[\thepage]{\emph{Disseny, programació i implementació d'un robot de dibuix amb Arduino}}



\section{Conclusions}



\section{Future work}
\setlength\topmargin{8mm}
\onehalfspacing
\chapter{Conclusions i treballs futurs} % Main chapter title

\label{Chapter8} % For referencing the chapter elsewhere, use \ref{Chapter1} 

\rhead[\emph{Disseny, programació i implementació d'un robot de dibuix amb Arduino}]{\thepage}
\lhead[\thepage]{\emph{Disseny, programació i implementació d'un robot de dibuix amb Arduino}}



\section{Conclusions}

Amb aquest projecte s'ha aconseguit crear un robot de dibuix passant per totes les etapes, des d'un estudi previ fins a la programació, passant pel disseny de l'estructura, el càlcul del moviment, la definició d'un mètode per convertir arxius SVG en trajectories i aquestes en ordres del robot, la implementació d'un sistema de comunicació i la la creació d'una aplicació que ho controli. Per tant, es considera que s'han assolit els objectius definits.

En primer lloc, s'ha estudiat la viabilitat de realitzar aquests robot com un sistema en llaç obert, en el qual els comportaments i el control dels diferents elements són bàsics per assegurar el correcte funcionament. S'ha pogut veure que, amb la tecnologia de la qual es disposava, s'ha pogut crear un prototip capaç de traçar figures de manera bastant aproximada, però que no es podira utilitzar per tasques de gran precisió. Per tal d'augmentar aquesta precisió es podrien avaluar dues vies diferents: en primer lloc una millora dels components i una feina de redisseny que donaria lloc a un robot de millors prestacions i amb les quals es podria reduir aquest error, però mai eliminar-lo del tot. En canvi, si s'opta per tancar el llaç de control amb altres dispositius, que poguessin mesurar i corregir l'error comés, es creu que si que es podria aconseguir un dispositiu d'alta precisió. Aquestes feines però quedaven fora de l'abast del projecte, i per tant es pot concloure és que sí és viable realitzar un robot en llaç obert tot i que la seva precisió sigui limitada. 

En quant al disseny i la construcció del robot s'ha assolit el nivell esperat, ja que s'ha dissenyat i construit l'estructura, s'han escollit els elements electrònics que el fan possible aconseguit realitzar aquest robot, amb les especificacions de tamany i funcionalitat desitjades. D'aquesta manera, s'ha complert amb aquest objectiu.

Per acabar, s'ha dotat el robot de tot el software necessari per tal de fer-lo funcionar, des de la transformació de dibuixos realitzats per l'usuari fins a ordres dels motors que s'envien per bluetooth des de l'ordinador i l'Arduino processa. Així doncs, s'han aconseguit assolir tots els objectius fins aconseguir la realització d'un robot de dibuix de manera molt satisfactoria.

Al llarg de tot el projecte no només s'han assolit els objectius i subobjectius, sinó que s'ha aprofundit en temes de programació en els llenguatges Python i Arduino, s'ha realitzat un disseny complet des de la idea fins a la realització d'un dispositiu que ha permés veure totes les etàpes de creació que hi ha darrere d'un producte, i s'ha posat en pràctica molts aspectes treballats durat el màster.


\section{Treballs futurs}
Aquest treball és una primera aproximació a la solució de la problematica presentada que pot donar lloc a futeres investigacions per tal d'assolir els reultats esperats. El robot creat, està dissenyat en llaç obert  i presenta una precisió limitada que inhabilita la seva utilització en algunes aplicacions i s'ha dissenyat per utiltzar un retolador i dibuixar sobre un paper  en una superficie plana. 

En primer lloc es podria estudiar la manera de millorar aquesta precisió en llaç obert per tal de mantenir el projecte com un sistema de baix cost fàcil de transoportar i utilitzar millorant els seus components i el seu disseny. 

Per altra banda, per assegurar el perfecte funcionament del mateix es podira dur a terme un estudi de les diferents vies possibles per tancar un llaç de control que permeti eliminar l'error i augementar la seva precisió. Per fer-ho es podrien estudiar mètodes que actuin de forma local al robot com seria l'utilització d'una IMU que pugui localitzar i orientar el robot, o mètodes externs com ara sistemes de visió o un sistema de localització i detecció de moviments per barres de dispositius LED infrarojos, per exemple. 

També seria interessant l'estudi de nous mètodes de funcionament a part de la utilització de retoladors, com podrien ser eines de tall per utilitzar sobre fusta o tela, guix per utilitzar sobre pissarres o el terra, pintura per tal de poder pintar sobre superficies planes o qualsevol altre aplicació que quelcom es pugui imaginar. Un altre camp d'estudi podria ser la superfície sobre la qual treballa, i poder-la adaptar a una superfície irregular o fins i tot mètodes per tal de poder treballar sobre parets verticals. 
\setlength\topmargin{8mm}
\onehalfspacing
\chapter{Conclusions i treballs futurs} % Main chapter title

\label{Chapter9} % For referencing the chapter elsewhere, use \ref{Chapter1} 

\rhead[\emph{Disseny, programació i implementació d'un robot de dibuix amb Arduino}]{\thepage}
\lhead[\thepage]{\emph{Disseny, programació i implementació d'un robot de dibuix amb Arduino}}



\section{Conclusions}

Amb aquest projecte s'ha aconseguit crear un robot de dibuix des d'un estudi previ fins a la programació, passant pel disseny de l'estructura, el càlcul del moviment, la definició d'un mètode per convertir arxius SVG en trajectòries i aquestes en ordres del robot, la implementació d'un sistema de comunicació i la creació d'una aplicació que ho controli. Per tant, es considera que s'han assolit els objectius definits.

En primer lloc, s'ha estudiat la viabilitat de realitzar aquest robot com un sistema en llaç obert, en el qual el comportament i el control dels diferents elements són bàsics per assegurar el correcte funcionament. S'ha pogut veure que, amb la tecnologia de la qual es disposava, s'ha pogut crear un prototip capaç de traçar figures de manera bastant aproximada, però que no es podria utilitzar per tasques de gran precisió. Per tal d'augmentar aquesta precisió es podrien avaluar dues vies diferents: en primer lloc, una millora dels components i una feina de redisseny que donaria lloc a un robot de millors prestacions i amb les quals es podria reduir aquest error, però mai eliminar-lo del tot. En canvi, si s'opta per tancar el llaç de control amb altres dispositius, que poguessin mesurar i corregir l'error comès, es creu que sí que es podria aconseguir un dispositiu d'alta precisió. Aquestes feines, però, quedaven fora de l'abast del projecte i ,per tant, es pot concloure que sí que és viable realitzar un robot en llaç obert tot i que la seva precisió sigui limitada. 

En quant al disseny i la construcció del robot s'ha assolit el nivell esperat, ja que s'ha dissenyat i construït l'estructura, s'han escollit els elements electrònics que fan possible la realització d'aquest robot, i s'han complert les especificacions de tamany i funcionalitat desitjades. D'aquesta manera, s'ha complert amb aquest objectiu.

Per acabar, s'ha dotat el robot de tot el software necessari per tal de fer-lo funcionar, des de la transformació de dibuixos realitzats per l'usuari fins a ordres dels motors, que s'envien per bluetooth des de l'ordinador i l'Arduino les processa. Així doncs, s'han aconseguit assolir tots els objectius fins aconseguir la realització d'un robot de dibuix de manera molt satisfactòria.

Al llarg de tot el projecte no només s'han assolit els objectius i subobjectius, sinó que s'ha aprofundit en temes de programació en els llenguatges Python i Arduino, s'ha realitzat un disseny complet des de la idea fins a la realització d'un dispositiu que ha permès veure totes les etapes de creació que hi ha darrere d'un producte, i s'han posat en pràctica molts aspectes treballats durant el màster.


\section{Possibles treballs futurs} \label{sec:treballsfuturs}
Aquest treball és una primera aproximació a la solució de la problemàtica presentada que pot donar lloc a futeres investigacions per tal d'assolir els resultats esperats. El robot creat està dissenyat en llaç obert  i presenta una precisió limitada que inhabilita la seva utilització en algunes aplicacions i s'ha dissenyat per utiltzar un retolador i dibuixar sobre un paper  en una superficie plana. 

En primer lloc, es podria estudiar la manera de millorar aquesta precisió en llaç obert per tal de mantenir el projecte com un sistema de baix cost, fàcil de transoportar i utilitzar millorant els seus components i el seu disseny. 

Per altra banda, per assegurar el perfecte funcionament del mateix es podria dur a terme un estudi de les diferents vies possibles per tancar un llaç de control que permeti eliminar l'error i augementar la seva precisió. Per fer-ho es podrien estudiar mètodes que actuin de forma local al robot com seria l'utilització d'una IMU que pugui localitzar i orientar el robot, o mètodes externs com ara sistemes de visió o un sistema de localització i detecció de moviments per barres de dispositius LED infrarojos, per exemple. 

També seria interessant l'estudi de nous mètodes de funcionament a part de la utilització de retoladors, com podrien ser eines de tall per utilitzar sobre fusta o tela, guix per utilitzar sobre pissarres o el terra, pintura per tal de poder pintar sobre superfícies planes o qualsevol altre aplicació que es pugui imaginar. Un altre camp d'estudi podria ser la superfície sobre la qual treballa, i poder-lo adaptar a una superfície irregular o fins i tot mètodes per tal de poder treballar sobre parets verticals. 




%----------------------------------------------------------------------------------------
%	THESIS CONTENT - APPENDICES
%----------------------------------------------------------------------------------------

\addtocontents{toc}{\vspace{2em}} % Add a gap in the Contents, for aesthetics

\appendix % Cue to tell LaTeX that the following 'chapters' are Appendices

% Include the appendices of the thesis as separate files from the Appendices folder
% Uncomment the lines as you write the Appendices

%% Appendix A

\chapter{Programes creats} % Main appendix title

\label{AppendixA} % For referencing this appendix elsewhere, use \ref{AppendixA}

\lhead{} % This is for the header on each page - perhaps a shortened title
\rhead{}


\definecolor{mygreen}{rgb}{0,0.6,0}
\definecolor{mygray}{rgb}{0.47,0.47,0.33}
\definecolor{myorange}{rgb}{0.8,0.4,0}
\definecolor{mywhite}{rgb}{0.98,0.98,0.98}
\definecolor{myblue}{rgb}{0.01,0.61,0.98}

%\newcommand*{\FormatDigit}[1]{\ttfamily\textcolor{mygreen}{#1}}
%% https://tex.stackexchange.com/questions/32174/listings-package-how-can-i-format-all-numbers
\lstdefinestyle{FormattedNumber}{%
	literate=*{0}{{\FormatDigit{0}}}{1}%
	{1}{{\FormatDigit{1}}}{1}%
	{2}{{\FormatDigit{2}}}{1}%
	{3}{{\FormatDigit{3}}}{1}%
	{4}{{\FormatDigit{4}}}{1}%
	{5}{{\FormatDigit{5}}}{1}%
	{6}{{\FormatDigit{6}}}{1}%
	{7}{{\FormatDigit{7}}}{1}%
	{8}{{\FormatDigit{8}}}{1}%
	{9}{{\FormatDigit{9}}}{1}%
	{.0}{{\FormatDigit{.0}}}{2}% Following is to ensure that only periods
	{.1}{{\FormatDigit{.1}}}{2}% followed by a digit are changed.
	{.2}{{\FormatDigit{.2}}}{2}%
	{.3}{{\FormatDigit{.3}}}{2}%
	{.4}{{\FormatDigit{.4}}}{2}%
	{.5}{{\FormatDigit{.5}}}{2}%
	{.6}{{\FormatDigit{.6}}}{2}%
	{.7}{{\FormatDigit{.7}}}{2}%
	{.8}{{\FormatDigit{.8}}}{2}%
	{.9}{{\FormatDigit{.9}}}{2}%
	%{,}{{\FormatDigit{,}}{1}% depends if you want the "," in color
	{\ }{{ }}{1}% handle the space
	,%
}


\lstset{%
	backgroundcolor=\color{mywhite},   
	basicstyle=\footnotesize,       
	breakatwhitespace=false,         
	breaklines=true,                 
	captionpos=b,                   
	commentstyle=\color{red},    
	deletekeywords={...},           
	escapeinside={\%*}{*)},          
	extendedchars=true,              
	%frame=shadowbox,                    
	keepspaces=true,                 
	keywordstyle=\color{myorange},       
	language=Octave,                
	morekeywords={*,...},            
	%numbers=left,                    
	%numbersep=5pt,                   
	%numberstyle=\tiny\color{mygray}, 
	rulecolor=\color{black},         
	rulesepcolor=\color{myblue},
	showspaces=false,                
	showstringspaces=false,          
	showtabs=false,                  
	%stepnumber=2,                    
	stringstyle=\color{myorange},    
	tabsize=2,                       
	title=\lstname,
	emphstyle=\bfseries\color{blue},%  style for emph={} 
}    

%% language specific settings:
\lstdefinestyle{Arduino}{%
	style=FormattedNumber,
	keywords={void, int, long, boolean, pinMode, digitalWrite},%                 define keywords
	morecomment=[l]{//},%             treat // as comments
	morecomment=[s]{/*}{*/},%         define /* ... */ comments
	emph={HIGH, OUTPUT, LOW},%        keywords to emphasize
}

% Afegir l'annex

\section{Python}

\subsection{RobotMoveBT.py}
\begin{python}
	import math, serial, time
	
	arduino=serial.Serial('COM4', 9600)
	time.sleep(0.1)
	
	StepsVolta=1600
	VelocitatMax=200
	x0 = 0.0
	y0 = 0.0
	#posicio[2];
	pasdreta=0; #posicio absoluta del motor dret en pasos
	pasesquerra=0; #posicio absoluta del motor dret en pasos
	DiamRoda=51.9; #diametre de la roda en mm
	RadiRoda=DiamRoda/2.0; #radi de la roda en mm
	d=122.0/2.0; #distancia en mm entre el centre de l'eix (punt del boli) i les rodes
	distDreta=61.8
	distEsquerra=61.9
	angle=0.0; #angle entre punt de la circumferencia en radians
	arc=0.0; #arc de circumferencia que s'ha de recorre en mm
	pas= math.pi *DiamRoda/StepsVolta; #mm recorreguts per pas del motor
	direccio0=math.pi/2; #angle inicial del robot a 90 graus
	tempsLectura=0.01; #temps per llegir l'Arduino
	
	
	def G00(x1,y1):
		global x0
		global y0
		global pasdreta
		global pasesquerra
		global distancia
		global pas
		global direccio0
		dx = x1 - x0;
		dy = y1 - y0;
		distancia = math.sqrt(dx*dx + dy*dy);
		steps=round(distancia/pas);
		if (dx != 0):
			direccio = math.atan(dy/dx);
			if (dx < 0 and dy >= 0):
				direccio = direccio + math.pi;
			elif (dx < 0 and dy < 0):
				direccio = direccio - math.pi
			apuntar(direccio);
			print (direccio)
		else:
			if (dy > 0):
				direccio = math.pi/2.0;
			elif (dy<0):
				direccio = -math.pi/2.0;
			else:
				direccio=direccio0
			apuntar(direccio);
		direccio0=direccio
		pasdreta=pasdreta+steps;
		pasesquerra=pasesquerra+steps;
		x0=x1;
		y0=y1;
		text='0'+','+str(pasdreta)+','+str(pasesquerra)+','
		arduino.write(text)
		robot=1
		while robot==1:
			if arduino.inWaiting()>0:
				st=arduino.readline().strip()
				time.sleep(tempsLectura)
				if st=='Ready':
					robot=0
		return
	
	def G01(x1,y1):
		global x0
		global y0
		global pasdreta
		global pasesquerra
		global distancia
		global pas
		dx = x1 - x0;
		dy = y1 - y0;
		distancia = math.sqrt(dx*dx + dy*dy);
		steps=round(distancia/pas);
		pasdreta=pasdreta+steps;
		pasesquerra=pasesquerra+steps;
		x0=x1;
		y0=y1;
		text='1'+','+str(pasdreta)+','+str(pasesquerra)+','
		arduino.write(text)
		robot=1
		while robot==1:
			if arduino.inWaiting()>0:
				st=arduino.readline().strip()
				time.sleep(tempsLectura)
				if st=='Ready':
					robot=0
		return
	
	def G02(x1,y1,xC,yC,direccio1):
		global x0
		global y0
		global pasdreta
		global pasesquerra
		global distancia
		global pas
		global direccio0
		RadiGirC=math.sqrt(xC*xC+yC*yC)
		xC=xC+x0;
		yC=yC+y0;
		x0=x0-xC;
		y0=y0-yC;
		x1=x1-xC;
		y1=y1-yC;
		angle0=math.atan2(y0,x0)
		angle1=math.atan2(y1,x1)
		angle=angle0-angle1
		if angle<0:
			angle=2*math.pi+angle
		distanciaD = angle * (RadiGirC-distDreta);
		distanciaE = angle * (RadiGirC+distEsquerra);
		stepsD=round(distanciaD/pas);
		stepsE=round(distanciaE/pas);
		pasdreta=pasdreta+stepsD;
		pasesquerra=pasesquerra+stepsE;
		x0=x1+xC;
		y0=y1+yC;
		direccio0=direccio1;
		text='1'+','+str(pasdreta)+','+str(pasesquerra)+','
		arduino.write(text)
		robot=1
		while robot==1:
			if arduino.inWaiting()>0:
				st=arduino.readline().strip()
				time.sleep(tempsLectura)
				if st=='Ready':
					robot=0
		return
	
	def G03(x1,y1,xC,yC,direccio1):
		global x0
		global y0
		global pasdreta
		global pasesquerra
		global distancia
		global pas
		global direccio0
		RadiGirC=math.sqrt(xC*xC+yC*yC)
		xC=xC+x0;
		yC=yC+y0;
		x0=x0-xC;
		y0=y0-yC;
		x1=x1-xC;
		y1=y1-yC;
		angle0=math.atan2(y0,x0)
		angle1=math.atan2(y1,x1)
		angle=angle1-angle0
		if angle<0:
			angle=2*math.pi+angle
		distanciaD = angle * (RadiGirC+distDreta);
		distanciaE = angle * (RadiGirC-distEsquerra);
		stepsD=round(distanciaD/pas);
		stepsE=round(distanciaE/pas);
		pasdreta=pasdreta+stepsD;
		pasesquerra=pasesquerra+stepsE;
		x0=x1+xC;
		y0=y1+yC;
		direccio0=direccio1;
		text='1'+','+str(pasdreta)+','+str(pasesquerra)+','
		arduino.write(text)
		robot=1
		while robot==1:
			if arduino.inWaiting()>0:
				st=arduino.readline().strip()
				time.sleep(tempsLectura)
				if st=='Ready':
					robot=0
		return
	
	def apuntar(direccio1):
		global pasdreta
		global pasesquerra
		global distancia
		global pas
		global direccio0
		gir=direccio0-direccio1;
		absgir=abs(gir);
		if (absgir > math.pi):
			if (gir<0):
				gir= 2.0 * math.pi + gir;
			else:
				gir= -2.0 * math.pi + gir;
		distancia=gir*d;
		steps = distancia / pas
		pasdreta=pasdreta-steps;
		pasesquerra=pasesquerra+steps;
		direccio0=direccio1;
		text='0'+','+str(pasdreta)+','+str(pasesquerra)+','
		arduino.write(text)
		robot=1
		while robot==1:
			if arduino.inWaiting()>0:
				st=arduino.readline().strip()
				time.sleep(tempsLectura)
				if st=='Ready':
					robot=0
		return
	
	
	def GCodeFile(oldfile, newfile):
		oldfile=open(oldfile, 'r')
		newfile=open(newfile, 'w')
		oldfile=oldfile.readlines()
		for linia in oldfile:
			linia=linia.strip()
			linia=linia.split()
			if linia!=[]:
				if linia[0]=='G00' and linia[1][0]=='X':
					x=float(linia[1][1:])
					y=float(linia[2][1:])
					G00(x,y)
					newfile.write('G00('+linia[1][1:]+', '+linia[2][1:]+');\n')
			
				elif linia[0]=='G01' and linia[1][0]=='X':
					x=float(linia[1][1:])
					y=float(linia[2][1:])
					G01(x,y)
					newfile.write('G01('+linia[1][1:]+', '+linia[2][1:]+');\n')
				
				elif linia[0]=='G01' and linia[1][0]=='A':
					angle=float(linia[1][1:])
					angle=angle-((2.0*math.pi)*(angle//(2.0*math.pi)))
					apuntar(angle)
					newfile.write('apuntar('+linia[1][1:]+');\n')
				
				elif linia[0]=='G02':
					x1=float(linia[1][1:])
					y1=float(linia[2][1:])
					xc=float(linia[4][1:])
					yc=float(linia[5][1:])
					angle=float(linia[-1][1:])
					angle=angle-((2.0*math.pi)*(angle//(2.0*math.pi)))
					G02(x1,y1,xc,yc,angle)
					newfile.write('G02('+linia[1][1:]+', '+linia[2][1:]+', '+linia[4][1:]+', '+linia[5][1:]+', '+linia[-1][1:]+');\n')
				
				elif linia[0]=='G03':
					x1=float(linia[1][1:])
					y1=float(linia[2][1:])
					xc=float(linia[4][1:])
					yc=float(linia[5][1:])
					angle=float(linia[-1][1:])
					angle=angle-((2.0*math.pi)*(angle//(2.0*math.pi)))
					G03(x1,y1,xc,yc,angle)
					newfile.write('G03('+linia[1][1:]+', '+linia[2][1:]+', '+linia[4][1:]+', '+linia[5][1:]+', '+linia[-1][1:]+');\n')
				
				else:
					pass
			else:
				pass
		newfile.close()
	
	
	def GCode(fitxer):
		fitxer=open(fitxer, 'r')
		fitxer=fitxer.readlines()
		for linia in fitxer:
			linia=linia.strip()
			linia=linia.split()
			if linia!=[]:
				if linia[0]=='G00' and linia[1][0]=='X':
					x=float(linia[1][1:])
					y=float(linia[2][1:])
					G00(x,y)
				
				
				elif linia[0]=='G01' and linia[1][0]=='X':
					x=float(linia[1][1:])
					y=float(linia[2][1:])
					G01(x,y)
				
				
				elif linia[0]=='G01' and linia[1][0]=='A':
					angle=float(linia[1][1:])
					angle=angle-((2.0*math.pi)*(angle//(2.0*math.pi)))
					apuntar(angle)
				
				
				elif linia[0]=='G02':
					x1=float(linia[1][1:])
					y1=float(linia[2][1:])
					xc=float(linia[4][1:])
					yc=float(linia[5][1:])
					angle=float(linia[-1][1:])
					angle=angle-((2.0*math.pi)*(angle//(2.0*math.pi)))
					G02(x1,y1,xc,yc,angle)
				
				
				elif linia[0]=='G03':
					x1=float(linia[1][1:])
					y1=float(linia[2][1:])
					xc=float(linia[4][1:])
					yc=float(linia[5][1:])
					angle=float(linia[-1][1:])
					angle=angle-((2.0*math.pi)*(angle//(2.0*math.pi)))
					G03(x1,y1,xc,yc,angle)
			
		
				else:
					pass
			else:
				pass
\end{python}

\newpage

\subsection{Draw.py}
\begin{python}
	import turtle
	import math
	import RobotMoveBT
	
	moviments=[]
	
	def cercle(t,R,angle):
		t.circle(R,angle)
	
	def recta(t,x,y):
		t.pendown()
		t.goto(x,y)
		
	def rectaup(t,x,y):
		t.penup()
		t.goto(x,y)
		
	def dibuixApuntar(t,angle):
		t.setheading(angle)
		ordre=['apuntar',angle]
		moviments.append(ordre)
	
	def dibuixG00(t,x1,y1):
		t.penup()
		(x0,y0)=t.pos()
		dx = x1 - x0;
		dy = y1 - y0
		if (dx != 0):
			direccio = math.atan(dy/dx);
			if (dx < 0 and dy >= 0):
				direccio = direccio + math.pi;
			elif (dx < 0 and dy < 0):
				direccio = direccio - math.pi
			t.setheading(direccio);
		else:
			if (dy > 0):
				direccio = math.pi/2.0;
			elif (dy<0):
				direccio = -math.pi/2.0;
			else:
				direccio=t.heading()
			t.setheading(direccio);
		t.goto(x1,y1)
		ordre=['G00',x1,y1]
		moviments.append(ordre)
		return
	
	def dibuixG01(t,x1,y1):
		t.pendown()
		(x0,y0)=t.pos()
		dx = x1 - x0;
		dy = y1 - y0
		if (dx != 0):
			direccio = math.atan(dy/dx);
			if (dx < 0 and dy >= 0):
				direccio = direccio + math.pi;
			elif (dx < 0 and dy < 0):
				direccio = direccio - math.pi
			t.setheading(direccio);
		else:
			if (dy > 0):
				direccio = math.pi/2.0;
			elif (dy<0):
				direccio = -math.pi/2.0;
			else:
				direccio=t.heading()
			t.setheading(direccio);
		t.goto(x1,y1)
		ordre=['apuntar',direccio]
		moviments.append(ordre)
		ordre=['G01',x1,y1]
		moviments.append(ordre)
		return
	
	
	def dibuixG02(t,x1,y1,xC,yC,angle):
		t.pendown()
		radi=math.sqrt(xC*xC + yC*yC)
		(x0,y0)=t.pos()
		xC=xC+x0;
		yC=yC+y0;
		x0=x0-xC;
		y0=y0-yC;
		x1=x1-xC;
		y1=y1-yC;
		angle0=math.atan2(y0,x0)
		angle1=math.atan2(y1,x1)
		angle=angle0-angle1
		if angle<0:
			angle=2*math.pi+angle
		t.circle(-radi,angle)
		ordre=['G02',x1,y1,xC,yC,t.heading()]
		moviments.append(ordre)
		return
	
	def dibuixG03(t,x1,y1,xC,yC,angle):
		t.pendown()
		radi=math.sqrt(xC*xC + yC*yC)
		(x0,y0)=t.pos()
		xC=xC+x0;
		yC=yC+y0;
		x0=x0-xC;
		y0=y0-yC;
		x1=x1-xC;
		y1=y1-yC;
		angle0=math.atan2(y0,x0)
		angle1=math.atan2(y1,x1)
		angle=angle1-angle0
		if angle<0:
			angle=2*math.pi+angle
		t.circle(radi,angle)
		ordre=['G03',x1,y1,xC,yC,t.heading()]
		moviments.append(ordre)
		return
	
	def afegir(t,val):
		moviments.append(val)
		return
	
	def desfer(t):
		t.undo()
		if moviments[-1][0]=='G01':
			del moviments[-1]
			del moviments[-1]
		else:
			del moviments[-1]
		return
	
	
	def previsualitzaFitxer(t,fitxer):
		fitxer=open(fitxer, 'r')
		fitxer=fitxer.readlines()
		for linia in fitxer:
			linia=linia.strip()
			linia=linia.split()
			if linia!=[]:
				if linia[0]=='G00' and linia[1][0]=='X':
					x=float(linia[1][1:])
					y=float(linia[2][1:])
					dibuixG00(t,x,y)
					
				
				elif linia[0]=='G01' and linia[1][0]=='X':
					x=float(linia[1][1:])
					y=float(linia[2][1:])
					dibuixG01(t,x,y)
					
				
				elif linia[0]=='G01' and linia[1][0]=='A':
					angle=float(linia[1][1:])
					angle=angle-((2.0*math.pi)*(angle//(2.0*math.pi)))
					dibuixApuntar(t,angle)
				
				
				elif linia[0]=='G02':
					x1=float(linia[1][1:])
					y1=float(linia[2][1:])
					xc=float(linia[4][1:])
					yc=float(linia[5][1:])
					angle=float(linia[-1][1:])
					angle=angle-((2.0*math.pi)*(angle//(2.0*math.pi)))
					dibuixG02(t,x1,y1,xc,yc,angle)
				
				
				elif linia[0]=='G03':
					x1=float(linia[1][1:])
					y1=float(linia[2][1:])
					xc=float(linia[4][1:])
					yc=float(linia[5][1:])
					angle=float(linia[-1][1:])
					angle=angle-((2.0*math.pi)*(angle//(2.0*math.pi)))
					dibuixG03(t,x1,y1,xc,yc,angle)
				
				
				else:
					pass
			else:
				pass
		return
	
	def reset():
		t.reset()
		return
	
	def representaLlist():
		for i in moviments:
			if i[0]=='G00':
				dibuixG00(i[1],i[2])
			elif i[0]=='G01':
				dibuixG01(i[1],i[2])
			elif i[0]=='G02':
				dibuixG02(i[1],i[2],i[3],i[4],i[5])
			elif i[0]=='G03':
				dibuixG03(i[1],i[2],i[3],i[4],i[5])
			elif i[0]=='apuntar':
				dibuixApuntar(i[1])
			else:
				pass
		return
	
	def dibuixaLlista():
		import RobotMoveBT
		for i in moviments:
			if i[0]=='G00':
				RobotMoveBT.G00(i[1],i[2])
			elif i[0]=='G01':
				RobotMoveBT.G01(i[1],i[2])
			elif i[0]=='G02':
				RobotMoveBT.G02(i[1],i[2],i[3],i[4],i[5])
			elif i[0]=='G03':
				RobotMoveBT.G03(i[1],i[2],i[3],i[4],i[5])
			elif i[0]=='apuntar':
				RobotMoveBT.apuntar(i[1])
			else:
				pass
		return
\end{python}

\newpage

\subsection{AppTFM.py}
\begin{python}
	import math
	
	import Tkinter as tk
	import ttk
	
	import turtle
	import Draw
	import RobotMoveBT
	
	LARGE_FONT= ("Verdana", 12)
	
	
	class TFM(tk.Tk):
	
		def __init__(self, *args, **kwargs):
		
			tk.Tk.__init__(self, *args, **kwargs)
			
			tk.Tk.wm_title(self, "TFM Josep Marti")
			
			
			container = tk.Frame(self)
			container.pack(side="top", fill="both", expand = True)
			container.grid_rowconfigure(0, weight=1)
			container.grid_columnconfigure(0, weight=1)
			
			self.frames = {}
			
			for F in (StartPage, Inkscape, Manual):
			
				frame = F(container, self)
				
				self.frames[F] = frame
				
				frame.grid(row=0, column=0, sticky="nsew")
				
			self.show_frame(StartPage)
		
		def show_frame(self, cont):
		
			frame = self.frames[cont]
			frame.tkraise()
		
	
	class StartPage(tk.Frame):
	
		def __init__(self, parent, controller):
			tk.Frame.__init__(self,parent)
			label = tk.Label(self, text="Com vols dibuixar?", font=LARGE_FONT)
			label.pack(pady=10,padx=10)
			
			button = ttk.Button(self, text="Importar archiu GCode creat per Inkscape",
			command=lambda: controller.show_frame(Inkscape))
			button.pack(pady=10,padx=10)
			
			button2 = ttk.Button(self, text="Realitzar un dibuix pas a pas",
			command=lambda: controller.show_frame(Manual))
			button2.pack(pady=5,padx=5)
	
	
	
	
	class Inkscape(tk.Frame):
	
		def __init__(self, parent, controller):
		nomfitxer=tk.StringVar(None)
		
		tk.Frame.__init__(self, parent)
		
		
		label = tk.Label(self, text="Importar archiu des de Inkscape", font=LARGE_FONT).pack(pady=10,padx=10)
		
		label2=tk.Label(self, text='Nom del fitxer:').pack()
		
		fitxer=tk.Entry(self, textvariable=nomfitxer).pack(pady=10,padx=10)
		
		button1 = ttk.Button(self, text="Dibuixar",command=lambda:Dibuixa(self)).pack(pady=10,padx=10)
		
		button2 = ttk.Button(self, text="Previsualitzar",command=lambda:Preview(self)).pack(pady=10,padx=10)
		
		button3 = ttk.Button(self, text="Tornar a l\'inici",command=lambda: back(self)).pack(pady=10,padx=10)
		
		canvas = tk.Canvas(self, width=500, height=500)
		canvas.pack(pady=10,padx=10)
		
		turtle1 = turtle.RawTurtle(canvas)
		turtle1.shape("turtle")
		turtle1.setheading(90)
		turtle1.radians()
		
		def Dibuixa(self):
			nom=nomfitxer.get()
			RobotMoveBT.GCode(nom)
		
		def Preview(self):
			nom=nomfitxer.get()
			Draw.previsualitzaFitxer(turtle1,nom)
		
		def back(self):
			turtle1.reset()
			turtle1.setheading(math.pi/2.0)
			controller.show_frame(StartPage)
	
	
	
	class Manual(tk.Frame):
	
		def __init__(self, parent, controller):
			
			tk.Frame.__init__(self, parent)
			
			label0=tk.Label(self, text='          ').grid(row=0,column=0)
			
			label = tk.Label(self, text="Pas a pas", font=LARGE_FONT).grid(row=1,column=2)
			
			label1=tk.Label(self, text='          ').grid(row=3,column=0)
			
			
			ordre=tk.StringVar()
			ordre.set(None)      
			
			G00X=tk.DoubleVar()
			G00Y=tk.DoubleVar()
			G01X=tk.DoubleVar()
			G01Y=tk.DoubleVar()
			G02X=tk.DoubleVar()
			G02Y=tk.DoubleVar()
			G02I=tk.DoubleVar()
			G02J=tk.DoubleVar()
			G02A=tk.DoubleVar()
			G03X=tk.DoubleVar()
			G03Y=tk.DoubleVar()
			G03I=tk.DoubleVar()
			G03J=tk.DoubleVar()
			G03A=tk.DoubleVar()
			RA=tk.DoubleVar()
			
			checkBox2Graus=tk.BooleanVar() 
			checkBox3Graus=tk.BooleanVar()
			checkBoxGraus=tk.BooleanVar()
			
			
			
			#G00
			radioG=tk.Radiobutton(self, text='Linia recta de posicionament (G00): ', value='G00', variable= ordre).grid(row=3,column=0)
			labelG00X=tk.Label(self, text='X =').grid(row=3,column=2)
			entryG00X=tk.Entry(self, textvariable= G00X, width=8).grid(row=3,column=3)
			labelG00Y=tk.Label(self, text='Y =').grid(row=3,column=5)
			entryG00Y=tk.Entry(self, textvariable=G00Y, width=8).grid(row=3,column=6)
			
			
			
			
			#G01
			radioG=tk.Radiobutton(self,text='Linia recta (G01): ', value='G01', variable=ordre).grid(row=5,column=0)
			labelG01X=tk.Label(self, text='X =').grid(row=5,column=2)
			entryG01X=tk.Entry(self, textvariable=G01X, width=8).grid(row=5,column=3)
			labelG01Y=tk.Label(self, text='Y =').grid(row=5,column=5)
			entryG01Y=tk.Entry(self, textvariable=G01Y, width=8).grid(row=5,column=6)
			
			
			
			
			#G02
			radioG=tk.Radiobutton(self,text='Arc en sentit horari (G02): ', value='G02', variable=ordre).grid(row=7,column=0)
			labelG02X=tk.Label(self, text='X =').grid(row=7,column=2)
			entryG02X=tk.Entry(self, textvariable=G02X, width=8).grid(row=7,column=3)
			labelG02Y=tk.Label(self, text='Y =').grid(row=7,column=5)
			entryG02Y=tk.Entry(self, textvariable=G02Y, width=8).grid(row=7,column=6)
			labelG02I=tk.Label(self, text='I =').grid(row=7,column=8)
			entryG02I=tk.Entry(self, textvariable=G02I, width=8).grid(row=7,column=9)
			labelG02J=tk.Label(self, text='J =').grid(row=7,column=11)
			entryG02J=tk.Entry(self, textvariable=G02J, width=8).grid(row=7,column=12)
			labelG02A=tk.Label(self, text='Orientacio =').grid(row=7,column=14)
			entryG02A=tk.Entry(self, textvariable=G02A, width=8).grid(row=7,column=15)
			
			
			
			
			
			
			#G03
			radioG=tk.Radiobutton(self,text='Arc en sentit horari (G03): ', value='G03', variable=ordre).grid(row=9,column=0)
			labelG03X=tk.Label(self, text='X =').grid(row=9,column=2)
			entryG03X=tk.Entry(self, textvariable=G03X, width=8).grid(row=9,column=3)
			labelG03Y=tk.Label(self, text='Y =').grid(row=9,column=5)
			entryG03Y=tk.Entry(self, textvariable=G03Y, width=8).grid(row=9,column=6)
			labelG03I=tk.Label(self, text='I =').grid(row=9,column=8)
			entryG03I=tk.Entry(self, textvariable=G03I, width=8).grid(row=9,column=9)
			labelG03J=tk.Label(self, text='J =').grid(row=9,column=11)
			entryG03J=tk.Entry(self, textvariable=G03J, width=8).grid(row=9,column=12)
			labelG03A=tk.Label(self, text='Orientacio =').grid(row=9,column=14)
			entryG03A=tk.Entry(self, textvariable=G03A, width=8).grid(row=9,column=15)
			
			
			
			
			
			#Rotacio
			radioG=tk.Radiobutton(self,text='Rotacio: ', value='apuntar', variable=ordre).grid(row=11,column=0)
			labelRA=tk.Label(self, text='Orientacio =').grid(row=11,column=2)
			entryRA=tk.Entry(self, textvariable=RA, width=8).grid(row=11,column=3)
			
			checkBox1=tk.Checkbutton(self, variable=checkBoxGraus, text="Orientacio en graus").grid(row=11, column=5)
			
			
			
			#Botons
			buttonDibuixar=tk.Button(self, text='Dibuixar', fg='blue', command=lambda:Dibuixar(self)).grid(row=20,column=7)
			buttonVeure=tk.Button(self, text='Veure', fg='blue', command=lambda: Veure(self)).grid(row=20,column=3)
			buttonUndo=tk.Button(self, text='Desfer', fg='blue', command=lambda: undo(self)).grid(row=20,column=5)
			buttonTorna=tk.Button(self, text='Tornar a l\'inici', fg='blue', command=lambda:back(self)).grid(row=20,column=9)
			
			canvas = tk.Canvas(self, width=500, height=500)
			canvas.grid(row=25,column=1,columnspan=12)
			
			turtle1 = turtle.RawTurtle(canvas)
			turtle1.shape("turtle")
			turtle1.setheading(90)
			turtle1.radians()
			
			label1=tk.Label(self, text='          ').grid(row=11,column=5)
			
			
			
			
			def Dibuixar(self):
				Draw.dibuixaLlista()
			
			
			def Veure(self):
				print ordre.get()
				tria=ordre.get()
				if tria == 'G00':
					x=G00X.get()
					y=G00Y.get()
					Draw.dibuixG00(turtle1,x,y)
				
				elif tria == 'G01':
					x=G01X.get()
					y=G01Y.get()
					Draw.dibuixG01(turtle1,x,y)
				
				elif tria == 'G02':
					x=G02X.get()
					y=G02Y.get()
					i=G02I.get()
					j=G02J.get()
					angle=G02A.get()
					graus=checkBox1.get()
					if graus == True:
						angle= (angle*math.pi)/180
					print (x, y, i , j, angle)
					Draw.dibuixG02(turtle1,x,y,i,j,angle)
				
				elif tria == 'G03':
					x=G03X.get()
					y=G03Y.get()
					i=G03I.get()
					j=G03J.get()
					angle=G03A.get()
					graus=checkBox1.get()
					if graus == True:
						angle= (angle*math.pi)/180 
					Draw.dibuixG03(turtle1,x,y,i,j,angle)
				
				elif tria == 'apuntar':
					angle=RA.get()
					graus=checkBox1.get()
					if graus == True:
						angle= (angle*math.pi)/180
					else:
						pass
					Draw.dibuixapuntar(turtle1,angle)
				
				else:
					pass
			
			def undo(self):
				Draw.desfer(turtle1)
			
			
			def back(self):
				turtle1.reset()
				turtle1.setheading(math.pi/2.0)
				controller.show_frame(StartPage)
	
	
	
	app = TFM()
	app.mainloop()
	
\end{python}

\section{Arduino}

\subsection{Robot.ino}

\begin{lstlisting}[style=Arduino]
	#include <AccelStepper.h>
	#include <MultiStepper.h>
	#include <Servo.h>
	#include <SoftwareSerial.h>
	
	SoftwareSerial bluetooth(10,11);
	
	Servo boli;
	int up=0;
	int down=50;
	
	AccelStepper RodaDreta(1,9,8);
	AccelStepper RodaEsquerra(1,13,12);
	MultiStepper Robot;
	int VelocitatMax=500;
	long posicio[2];
	int pasdreta=0; //posicio absoluta del motor dret en pasos
	int pasesquerra=0; //posicio absoluta del motor dret en pasos
	
	int text[3]; 
	int cnt=0;
	boolean Rebut = false;
	
	
	
	
	
	void setup() {
		bluetooth.begin(9600);
		
		
		pinMode(2, OUTPUT); //microstepping off
		pinMode(3,OUTPUT);
		digitalWrite(2,HIGH);
		digitalWrite(3,HIGH);
		pinMode(4, OUTPUT);
		pinMode(5,OUTPUT);
		digitalWrite(4,HIGH);
		digitalWrite(5,HIGH);
		boli.attach(6);
		boli.write(0);
		
		Robot.addStepper(RodaDreta);
		Robot.addStepper(RodaEsquerra);
		
		RodaDreta.setMaxSpeed(VelocitatMax);
		RodaEsquerra.setMaxSpeed(VelocitatMax);
	}
	
	void loop() {
		getSerialData();
		delay(1);
		processData();
	}
	
	void getSerialData(){
		if(bluetooth.available() > 0) {
			String x = bluetooth.readString();
			String buff="";
			
			for (int i=0; i<x.length();i++){
				String caracter="";
				caracter=caracter+x[i];
				if (caracter!=","){
					buff=buff+x[i];
				}
				else {
					int y=buff.toInt();
					text[cnt]=y;
					buff="";
					if (cnt<3){
						cnt+=1;
						}
					if (cnt==3){
						cnt=0;
					}
				}
			}
			Rebut=true;
			delay(1);    
		}
	}
	
	
	void processData(){
		if (Rebut==true){
			if (text[0]==0 and boli.read()!=up){
				boli.write(up);
				delay(300);
			}
			else if (text[0]==1 and boli.read()!=down){
				boli.write(down);
				delay(300);
			}
			posicio[0]=-text[1];
			posicio[1]=text[2];
			Robot.moveTo(posicio);
			Robot.runSpeedToPosition();
			bluetooth.println("Ready");
			Rebut=false; 
		}
	}
\end{lstlisting}
%% Appendix A

\chapter{Plànols de les peces} % Main appendix title

\label{AppendixB} % For referencing this appendix elsewhere, use \ref{AppendixA}

\lhead{} % This is for the header on each page - perhaps a shortened title
\rhead{}

\textbf{B.1 Xassís}\\

\textbf{B.2 Guia del retolador}\\

\textbf{B.3 Mecanisme d'elevació del retolador}\\

\textbf{B.4 Roda boja davantera}\\

\textbf{B.5 Roda motriu}\\

\textbf{B.6 Tubs auxiliars}\\




\newpage
\section{Xassís} 
\begin{picture} (0,0)
	\put(-82,-722){\includegraphics{XassisPlanol}}
\end{picture}

\newpage
\section{Guia del retolador} 
\begin{picture} (0,0)
\put(-105,-722){\includegraphics{GuiaPlanol}}
\end{picture}

\newpage
\section{Mecanisme d'elevació del retolador} 
\begin{picture} (0,0)
\put(-82,-722){\includegraphics{SuportPlanol}}
\end{picture}

\newpage
\section{Roda boja davantera} 
\begin{picture} (0,0)
\put(-105,-722){\includegraphics{RodaBojaPlanol}}
\end{picture}

\newpage
\section{Roda motriu} 
\begin{picture} (0,0)
\put(-82,-722){\includegraphics{RodaPlanol}}
\end{picture}

\newpage
\section{Tubs auxiliars} 
\begin{picture} (0,0)
\put(-105,-722){\includegraphics{TubsPlanol}}
\end{picture}

%\includepdf[page={1},fitpaper=true, trim=0mm 0mm 0mm 0mm, clip]{GuiaPlanol}
%\input{Appendices/AppendixC}

\addtocontents{toc}{\vspace{2em}} % Add a gap in the Contents, for aesthetics

\backmatter

%----------------------------------------------------------------------------------------
%	BIBLIOGRAPHY
%----------------------------------------------------------------------------------------

%\label{Bibliography}

%\lhead{\emph{Bibliography}} % Change the page header to say "Bibliography"

%\bibliographystyle{unsrtnat} % Use the "unsrtnat" BibTeX style for formatting the Bibliography

%\bibliography{Bibliography} % The references (bibliography) information are stored in the file named "Bibliography.bib"

\bibliographystyle{plain}
\bibliography{Bibliography}

\end{document}  
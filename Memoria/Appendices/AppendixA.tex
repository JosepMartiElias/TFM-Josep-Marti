% Appendix A

\chapter{Programes creats} % Main appendix title

\label{AppendixA} % For referencing this appendix elsewhere, use \ref{AppendixA}

\lhead{} % This is for the header on each page - perhaps a shortened title
\rhead{}


\definecolor{mygreen}{rgb}{0,0.6,0}
\definecolor{mygray}{rgb}{0.47,0.47,0.33}
\definecolor{myorange}{rgb}{0.8,0.4,0}
\definecolor{mywhite}{rgb}{0.98,0.98,0.98}
\definecolor{myblue}{rgb}{0.01,0.61,0.98}

%\newcommand*{\FormatDigit}[1]{\ttfamily\textcolor{mygreen}{#1}}
%% https://tex.stackexchange.com/questions/32174/listings-package-how-can-i-format-all-numbers
\lstdefinestyle{FormattedNumber}{%
	literate=*{0}{{\FormatDigit{0}}}{1}%
	{1}{{\FormatDigit{1}}}{1}%
	{2}{{\FormatDigit{2}}}{1}%
	{3}{{\FormatDigit{3}}}{1}%
	{4}{{\FormatDigit{4}}}{1}%
	{5}{{\FormatDigit{5}}}{1}%
	{6}{{\FormatDigit{6}}}{1}%
	{7}{{\FormatDigit{7}}}{1}%
	{8}{{\FormatDigit{8}}}{1}%
	{9}{{\FormatDigit{9}}}{1}%
	{.0}{{\FormatDigit{.0}}}{2}% Following is to ensure that only periods
	{.1}{{\FormatDigit{.1}}}{2}% followed by a digit are changed.
	{.2}{{\FormatDigit{.2}}}{2}%
	{.3}{{\FormatDigit{.3}}}{2}%
	{.4}{{\FormatDigit{.4}}}{2}%
	{.5}{{\FormatDigit{.5}}}{2}%
	{.6}{{\FormatDigit{.6}}}{2}%
	{.7}{{\FormatDigit{.7}}}{2}%
	{.8}{{\FormatDigit{.8}}}{2}%
	{.9}{{\FormatDigit{.9}}}{2}%
	%{,}{{\FormatDigit{,}}{1}% depends if you want the "," in color
	{\ }{{ }}{1}% handle the space
	,%
}


\lstset{%
	backgroundcolor=\color{mywhite},   
	basicstyle=\footnotesize,       
	breakatwhitespace=false,         
	breaklines=true,                 
	captionpos=b,                   
	commentstyle=\color{red},    
	deletekeywords={...},           
	escapeinside={\%*}{*)},          
	extendedchars=true,              
	%frame=shadowbox,                    
	keepspaces=true,                 
	keywordstyle=\color{myorange},       
	language=Octave,                
	morekeywords={*,...},            
	%numbers=left,                    
	%numbersep=5pt,                   
	%numberstyle=\tiny\color{mygray}, 
	rulecolor=\color{black},         
	rulesepcolor=\color{myblue},
	showspaces=false,                
	showstringspaces=false,          
	showtabs=false,                  
	%stepnumber=2,                    
	stringstyle=\color{myorange},    
	tabsize=2,                       
	title=\lstname,
	emphstyle=\bfseries\color{blue},%  style for emph={} 
}    

%% language specific settings:
\lstdefinestyle{Arduino}{%
	style=FormattedNumber,
	keywords={void, int, long, boolean, pinMode, digitalWrite},%                 define keywords
	morecomment=[l]{//},%             treat // as comments
	morecomment=[s]{/*}{*/},%         define /* ... */ comments
	emph={HIGH, OUTPUT, LOW},%        keywords to emphasize
}

% Afegir l'annex

\section{Python}

\subsection{RobotMoveBT.py} \label{PyRobot}
\begin{python}
	import math, serial, time
	
	arduino=serial.Serial('COM4', 9600)
	time.sleep(0.1)
	
	StepsVolta=1600
	VelocitatMax=200
	x0 = 0.0
	y0 = 0.0
	#posicio[2];
	pasdreta=0; #posicio absoluta del motor dret en pasos
	pasesquerra=0; #posicio absoluta del motor dret en pasos
	DiamRoda=51.9; #diametre de la roda en mm
	RadiRoda=DiamRoda/2.0; #radi de la roda en mm
	d=122.0/2.0; #distancia en mm entre el centre de l'eix (punt del boli) i les rodes
	distDreta=61.8
	distEsquerra=61.9
	angle=0.0; #angle entre punt de la circumferencia en radians
	arc=0.0; #arc de circumferencia que s'ha de recorre en mm
	pas= math.pi *DiamRoda/StepsVolta; #mm recorreguts per pas del motor
	direccio0=math.pi/2; #angle inicial del robot a 90 graus
	tempsLectura=0.01; #temps per llegir l'Arduino
	
	
	def G00(x1,y1):
		global x0
		global y0
		global pasdreta
		global pasesquerra
		global distancia
		global pas
		global direccio0
		dx = x1 - x0;
		dy = y1 - y0;
		distancia = math.sqrt(dx*dx + dy*dy);
		steps=round(distancia/pas);
		if (dx != 0):
			direccio = math.atan(dy/dx);
			if (dx < 0 and dy >= 0):
				direccio = direccio + math.pi;
			elif (dx < 0 and dy < 0):
				direccio = direccio - math.pi
			apuntar(direccio);
			print (direccio)
		else:
			if (dy > 0):
				direccio = math.pi/2.0;
			elif (dy<0):
				direccio = -math.pi/2.0;
			else:
				direccio=direccio0
			apuntar(direccio);
		direccio0=direccio
		pasdreta=pasdreta+steps;
		pasesquerra=pasesquerra+steps;
		x0=x1;
		y0=y1;
		text='0'+','+str(pasdreta)+','+str(pasesquerra)+','
		arduino.write(text)
		robot=1
		while robot==1:
			if arduino.inWaiting()>0:
				st=arduino.readline().strip()
				time.sleep(tempsLectura)
				if st=='Ready':
					robot=0
		return
	
	def G01(x1,y1):
		global x0
		global y0
		global pasdreta
		global pasesquerra
		global distancia
		global pas
		dx = x1 - x0;
		dy = y1 - y0;
		distancia = math.sqrt(dx*dx + dy*dy);
		steps=round(distancia/pas);
		pasdreta=pasdreta+steps;
		pasesquerra=pasesquerra+steps;
		x0=x1;
		y0=y1;
		text='1'+','+str(pasdreta)+','+str(pasesquerra)+','
		arduino.write(text)
		robot=1
		while robot==1:
			if arduino.inWaiting()>0:
				st=arduino.readline().strip()
				time.sleep(tempsLectura)
				if st=='Ready':
					robot=0
		return
	
	def G02(x1,y1,xC,yC,direccio1):
		global x0
		global y0
		global pasdreta
		global pasesquerra
		global distancia
		global pas
		global direccio0
		RadiGirC=math.sqrt(xC*xC+yC*yC)
		xC=xC+x0;
		yC=yC+y0;
		x0=x0-xC;
		y0=y0-yC;
		x1=x1-xC;
		y1=y1-yC;
		angle0=math.atan2(y0,x0)
		angle1=math.atan2(y1,x1)
		angle=angle0-angle1
		if angle<0:
			angle=2*math.pi+angle
		distanciaD = angle * (RadiGirC-distDreta);
		distanciaE = angle * (RadiGirC+distEsquerra);
		stepsD=round(distanciaD/pas);
		stepsE=round(distanciaE/pas);
		pasdreta=pasdreta+stepsD;
		pasesquerra=pasesquerra+stepsE;
		x0=x1+xC;
		y0=y1+yC;
		direccio0=direccio1;
		text='1'+','+str(pasdreta)+','+str(pasesquerra)+','
		arduino.write(text)
		robot=1
		while robot==1:
			if arduino.inWaiting()>0:
				st=arduino.readline().strip()
				time.sleep(tempsLectura)
				if st=='Ready':
					robot=0
		return
	
	def G03(x1,y1,xC,yC,direccio1):
		global x0
		global y0
		global pasdreta
		global pasesquerra
		global distancia
		global pas
		global direccio0
		RadiGirC=math.sqrt(xC*xC+yC*yC)
		xC=xC+x0;
		yC=yC+y0;
		x0=x0-xC;
		y0=y0-yC;
		x1=x1-xC;
		y1=y1-yC;
		angle0=math.atan2(y0,x0)
		angle1=math.atan2(y1,x1)
		angle=angle1-angle0
		if angle<0:
			angle=2*math.pi+angle
		distanciaD = angle * (RadiGirC+distDreta);
		distanciaE = angle * (RadiGirC-distEsquerra);
		stepsD=round(distanciaD/pas);
		stepsE=round(distanciaE/pas);
		pasdreta=pasdreta+stepsD;
		pasesquerra=pasesquerra+stepsE;
		x0=x1+xC;
		y0=y1+yC;
		direccio0=direccio1;
		text='1'+','+str(pasdreta)+','+str(pasesquerra)+','
		arduino.write(text)
		robot=1
		while robot==1:
			if arduino.inWaiting()>0:
				st=arduino.readline().strip()
				time.sleep(tempsLectura)
				if st=='Ready':
					robot=0
		return
	
	def apuntar(direccio1):
		global pasdreta
		global pasesquerra
		global distancia
		global pas
		global direccio0
		gir=direccio0-direccio1;
		absgir=abs(gir);
		if (absgir > math.pi):
			if (gir<0):
				gir= 2.0 * math.pi + gir;
			else:
				gir= -2.0 * math.pi + gir;
		distancia=gir*d;
		steps = distancia / pas
		pasdreta=pasdreta-steps;
		pasesquerra=pasesquerra+steps;
		direccio0=direccio1;
		text='0'+','+str(pasdreta)+','+str(pasesquerra)+','
		arduino.write(text)
		robot=1
		while robot==1:
			if arduino.inWaiting()>0:
				st=arduino.readline().strip()
				time.sleep(tempsLectura)
				if st=='Ready':
					robot=0
		return
	
	
	def GCodeFile(oldfile, newfile):
		oldfile=open(oldfile, 'r')
		newfile=open(newfile, 'w')
		oldfile=oldfile.readlines()
		for linia in oldfile:
			linia=linia.strip()
			linia=linia.split()
			if linia!=[]:
				if linia[0]=='G00' and linia[1][0]=='X':
					x=float(linia[1][1:])
					y=float(linia[2][1:])
					G00(x,y)
					newfile.write('G00('+linia[1][1:]+', '+linia[2][1:]+');\n')
			
				elif linia[0]=='G01' and linia[1][0]=='X':
					x=float(linia[1][1:])
					y=float(linia[2][1:])
					G01(x,y)
					newfile.write('G01('+linia[1][1:]+', '+linia[2][1:]+');\n')
				
				elif linia[0]=='G01' and linia[1][0]=='A':
					angle=float(linia[1][1:])
					angle=angle-((2.0*math.pi)*(angle//(2.0*math.pi)))
					apuntar(angle)
					newfile.write('apuntar('+linia[1][1:]+');\n')
				
				elif linia[0]=='G02':
					x1=float(linia[1][1:])
					y1=float(linia[2][1:])
					xc=float(linia[4][1:])
					yc=float(linia[5][1:])
					angle=float(linia[-1][1:])
					angle=angle-((2.0*math.pi)*(angle//(2.0*math.pi)))
					G02(x1,y1,xc,yc,angle)
					newfile.write('G02('+linia[1][1:]+', '+linia[2][1:]+', '+linia[4][1:]+', '+linia[5][1:]+', '+linia[-1][1:]+');\n')
				
				elif linia[0]=='G03':
					x1=float(linia[1][1:])
					y1=float(linia[2][1:])
					xc=float(linia[4][1:])
					yc=float(linia[5][1:])
					angle=float(linia[-1][1:])
					angle=angle-((2.0*math.pi)*(angle//(2.0*math.pi)))
					G03(x1,y1,xc,yc,angle)
					newfile.write('G03('+linia[1][1:]+', '+linia[2][1:]+', '+linia[4][1:]+', '+linia[5][1:]+', '+linia[-1][1:]+');\n')
				
				else:
					pass
			else:
				pass
		newfile.close()
	
	
	def GCode(fitxer):
		fitxer=open(fitxer, 'r')
		fitxer=fitxer.readlines()
		for linia in fitxer:
			linia=linia.strip()
			linia=linia.split()
			if linia!=[]:
				if linia[0]=='G00' and linia[1][0]=='X':
					x=float(linia[1][1:])
					y=float(linia[2][1:])
					G00(x,y)
				
				
				elif linia[0]=='G01' and linia[1][0]=='X':
					x=float(linia[1][1:])
					y=float(linia[2][1:])
					G01(x,y)
				
				
				elif linia[0]=='G01' and linia[1][0]=='A':
					angle=float(linia[1][1:])
					angle=angle-((2.0*math.pi)*(angle//(2.0*math.pi)))
					apuntar(angle)
				
				
				elif linia[0]=='G02':
					x1=float(linia[1][1:])
					y1=float(linia[2][1:])
					xc=float(linia[4][1:])
					yc=float(linia[5][1:])
					angle=float(linia[-1][1:])
					angle=angle-((2.0*math.pi)*(angle//(2.0*math.pi)))
					G02(x1,y1,xc,yc,angle)
				
				
				elif linia[0]=='G03':
					x1=float(linia[1][1:])
					y1=float(linia[2][1:])
					xc=float(linia[4][1:])
					yc=float(linia[5][1:])
					angle=float(linia[-1][1:])
					angle=angle-((2.0*math.pi)*(angle//(2.0*math.pi)))
					G03(x1,y1,xc,yc,angle)
			
		
				else:
					pass
			else:
				pass
\end{python}

\newpage

\subsection{Draw.py} \label{PyDraw}
\begin{python}
	import turtle
	import math
	import RobotMoveBT
	
	moviments=[]
	
	def cercle(t,R,angle):
		t.circle(R,angle)
	
	def recta(t,x,y):
		t.pendown()
		t.goto(x,y)
		
	def rectaup(t,x,y):
		t.penup()
		t.goto(x,y)
		
	def dibuixApuntar(t,angle):
		t.setheading(angle)
		ordre=['apuntar',angle]
		moviments.append(ordre)
	
	def dibuixG00(t,x1,y1):
		t.penup()
		(x0,y0)=t.pos()
		dx = x1 - x0;
		dy = y1 - y0
		if (dx != 0):
			direccio = math.atan(dy/dx);
			if (dx < 0 and dy >= 0):
				direccio = direccio + math.pi;
			elif (dx < 0 and dy < 0):
				direccio = direccio - math.pi
			t.setheading(direccio);
		else:
			if (dy > 0):
				direccio = math.pi/2.0;
			elif (dy<0):
				direccio = -math.pi/2.0;
			else:
				direccio=t.heading()
			t.setheading(direccio);
		t.goto(x1,y1)
		ordre=['G00',x1,y1]
		moviments.append(ordre)
		return
	
	def dibuixG01(t,x1,y1):
		t.pendown()
		(x0,y0)=t.pos()
		dx = x1 - x0;
		dy = y1 - y0
		if (dx != 0):
			direccio = math.atan(dy/dx);
			if (dx < 0 and dy >= 0):
				direccio = direccio + math.pi;
			elif (dx < 0 and dy < 0):
				direccio = direccio - math.pi
			t.setheading(direccio);
		else:
			if (dy > 0):
				direccio = math.pi/2.0;
			elif (dy<0):
				direccio = -math.pi/2.0;
			else:
				direccio=t.heading()
			t.setheading(direccio);
		t.goto(x1,y1)
		ordre=['apuntar',direccio]
		moviments.append(ordre)
		ordre=['G01',x1,y1]
		moviments.append(ordre)
		return
	
	
	def dibuixG02(t,x1,y1,xC,yC,angle):
		t.pendown()
		radi=math.sqrt(xC*xC + yC*yC)
		(x0,y0)=t.pos()
		xC=xC+x0;
		yC=yC+y0;
		x0=x0-xC;
		y0=y0-yC;
		x1=x1-xC;
		y1=y1-yC;
		angle0=math.atan2(y0,x0)
		angle1=math.atan2(y1,x1)
		angle=angle0-angle1
		if angle<0:
			angle=2*math.pi+angle
		t.circle(-radi,angle)
		ordre=['G02',x1,y1,xC,yC,t.heading()]
		moviments.append(ordre)
		return
	
	def dibuixG03(t,x1,y1,xC,yC,angle):
		t.pendown()
		radi=math.sqrt(xC*xC + yC*yC)
		(x0,y0)=t.pos()
		xC=xC+x0;
		yC=yC+y0;
		x0=x0-xC;
		y0=y0-yC;
		x1=x1-xC;
		y1=y1-yC;
		angle0=math.atan2(y0,x0)
		angle1=math.atan2(y1,x1)
		angle=angle1-angle0
		if angle<0:
			angle=2*math.pi+angle
		t.circle(radi,angle)
		ordre=['G03',x1,y1,xC,yC,t.heading()]
		moviments.append(ordre)
		return
	
	def afegir(t,val):
		moviments.append(val)
		return
	
	def desfer(t):
		t.undo()
		if moviments[-1][0]=='G01':
			del moviments[-1]
			del moviments[-1]
		else:
			del moviments[-1]
		return
	
	
	def previsualitzaFitxer(t,fitxer):
		fitxer=open(fitxer, 'r')
		fitxer=fitxer.readlines()
		for linia in fitxer:
			linia=linia.strip()
			linia=linia.split()
			if linia!=[]:
				if linia[0]=='G00' and linia[1][0]=='X':
					x=float(linia[1][1:])
					y=float(linia[2][1:])
					dibuixG00(t,x,y)
					
				
				elif linia[0]=='G01' and linia[1][0]=='X':
					x=float(linia[1][1:])
					y=float(linia[2][1:])
					dibuixG01(t,x,y)
					
				
				elif linia[0]=='G01' and linia[1][0]=='A':
					angle=float(linia[1][1:])
					angle=angle-((2.0*math.pi)*(angle//(2.0*math.pi)))
					dibuixApuntar(t,angle)
				
				
				elif linia[0]=='G02':
					x1=float(linia[1][1:])
					y1=float(linia[2][1:])
					xc=float(linia[4][1:])
					yc=float(linia[5][1:])
					angle=float(linia[-1][1:])
					angle=angle-((2.0*math.pi)*(angle//(2.0*math.pi)))
					dibuixG02(t,x1,y1,xc,yc,angle)
				
				
				elif linia[0]=='G03':
					x1=float(linia[1][1:])
					y1=float(linia[2][1:])
					xc=float(linia[4][1:])
					yc=float(linia[5][1:])
					angle=float(linia[-1][1:])
					angle=angle-((2.0*math.pi)*(angle//(2.0*math.pi)))
					dibuixG03(t,x1,y1,xc,yc,angle)
				
				
				else:
					pass
			else:
				pass
		return
	
	def reset():
		t.reset()
		return
	
	def representaLlist():
		for i in moviments:
			if i[0]=='G00':
				dibuixG00(i[1],i[2])
			elif i[0]=='G01':
				dibuixG01(i[1],i[2])
			elif i[0]=='G02':
				dibuixG02(i[1],i[2],i[3],i[4],i[5])
			elif i[0]=='G03':
				dibuixG03(i[1],i[2],i[3],i[4],i[5])
			elif i[0]=='apuntar':
				dibuixApuntar(i[1])
			else:
				pass
		return
	
	def dibuixaLlista():
		import RobotMoveBT
		for i in moviments:
			if i[0]=='G00':
				RobotMoveBT.G00(i[1],i[2])
			elif i[0]=='G01':
				RobotMoveBT.G01(i[1],i[2])
			elif i[0]=='G02':
				RobotMoveBT.G02(i[1],i[2],i[3],i[4],i[5])
			elif i[0]=='G03':
				RobotMoveBT.G03(i[1],i[2],i[3],i[4],i[5])
			elif i[0]=='apuntar':
				RobotMoveBT.apuntar(i[1])
			else:
				pass
		return
\end{python}

\newpage

\subsection{AppTFM.py} \label{PyApp}
\begin{python}
	import math
	
	import Tkinter as tk
	import ttk
	
	import turtle
	import Draw
	import RobotMoveBT
	
	LARGE_FONT= ("Verdana", 12)
	
	
	class TFM(tk.Tk):
	
		def __init__(self, *args, **kwargs):
		
			tk.Tk.__init__(self, *args, **kwargs)
			
			tk.Tk.wm_title(self, "TFM Josep Marti")
			
			
			container = tk.Frame(self)
			container.pack(side="top", fill="both", expand = True)
			container.grid_rowconfigure(0, weight=1)
			container.grid_columnconfigure(0, weight=1)
			
			self.frames = {}
			
			for F in (StartPage, Inkscape, Manual):
			
				frame = F(container, self)
				
				self.frames[F] = frame
				
				frame.grid(row=0, column=0, sticky="nsew")
				
			self.show_frame(StartPage)
		
		def show_frame(self, cont):
		
			frame = self.frames[cont]
			frame.tkraise()
		
	
	class StartPage(tk.Frame):
	
		def __init__(self, parent, controller):
			tk.Frame.__init__(self,parent)
			label = tk.Label(self, text="Com vols dibuixar?", font=LARGE_FONT)
			label.pack(pady=10,padx=10)
			
			button = ttk.Button(self, text="Importar archiu GCode creat per Inkscape",
			command=lambda: controller.show_frame(Inkscape))
			button.pack(pady=10,padx=10)
			
			button2 = ttk.Button(self, text="Realitzar un dibuix pas a pas",
			command=lambda: controller.show_frame(Manual))
			button2.pack(pady=5,padx=5)
	
	
	
	
	class Inkscape(tk.Frame):
	
		def __init__(self, parent, controller):
		nomfitxer=tk.StringVar(None)
		
		tk.Frame.__init__(self, parent)
		
		
		label = tk.Label(self, text="Importar archiu des de Inkscape", font=LARGE_FONT).pack(pady=10,padx=10)
		
		label2=tk.Label(self, text='Nom del fitxer:').pack()
		
		fitxer=tk.Entry(self, textvariable=nomfitxer).pack(pady=10,padx=10)
		
		button1 = ttk.Button(self, text="Dibuixar",command=lambda:Dibuixa(self)).pack(pady=10,padx=10)
		
		button2 = ttk.Button(self, text="Previsualitzar",command=lambda:Preview(self)).pack(pady=10,padx=10)
		
		button3 = ttk.Button(self, text="Tornar a l\'inici",command=lambda: back(self)).pack(pady=10,padx=10)
		
		canvas = tk.Canvas(self, width=500, height=500)
		canvas.pack(pady=10,padx=10)
		
		turtle1 = turtle.RawTurtle(canvas)
		turtle1.shape("turtle")
		turtle1.setheading(90)
		turtle1.radians()
		
		def Dibuixa(self):
			nom=nomfitxer.get()
			RobotMoveBT.GCode(nom)
		
		def Preview(self):
			nom=nomfitxer.get()
			Draw.previsualitzaFitxer(turtle1,nom)
		
		def back(self):
			turtle1.reset()
			turtle1.setheading(math.pi/2.0)
			controller.show_frame(StartPage)
	
	
	
	class Manual(tk.Frame):
	
		def __init__(self, parent, controller):
			
			tk.Frame.__init__(self, parent)
			
			label0=tk.Label(self, text='          ').grid(row=0,column=0)
			
			label = tk.Label(self, text="Pas a pas", font=LARGE_FONT).grid(row=1,column=2)
			
			label1=tk.Label(self, text='          ').grid(row=3,column=0)
			
			
			ordre=tk.StringVar()
			ordre.set(None)      
			
			G00X=tk.DoubleVar()
			G00Y=tk.DoubleVar()
			G01X=tk.DoubleVar()
			G01Y=tk.DoubleVar()
			G02X=tk.DoubleVar()
			G02Y=tk.DoubleVar()
			G02I=tk.DoubleVar()
			G02J=tk.DoubleVar()
			G02A=tk.DoubleVar()
			G03X=tk.DoubleVar()
			G03Y=tk.DoubleVar()
			G03I=tk.DoubleVar()
			G03J=tk.DoubleVar()
			G03A=tk.DoubleVar()
			RA=tk.DoubleVar()
			
			checkBox2Graus=tk.BooleanVar() 
			checkBox3Graus=tk.BooleanVar()
			checkBoxGraus=tk.BooleanVar()
			
			
			
			#G00
			radioG=tk.Radiobutton(self, text='Linia recta de posicionament (G00): ', value='G00', variable= ordre).grid(row=3,column=0)
			labelG00X=tk.Label(self, text='X =').grid(row=3,column=2)
			entryG00X=tk.Entry(self, textvariable= G00X, width=8).grid(row=3,column=3)
			labelG00Y=tk.Label(self, text='Y =').grid(row=3,column=5)
			entryG00Y=tk.Entry(self, textvariable=G00Y, width=8).grid(row=3,column=6)
			
			
			
			
			#G01
			radioG=tk.Radiobutton(self,text='Linia recta (G01): ', value='G01', variable=ordre).grid(row=5,column=0)
			labelG01X=tk.Label(self, text='X =').grid(row=5,column=2)
			entryG01X=tk.Entry(self, textvariable=G01X, width=8).grid(row=5,column=3)
			labelG01Y=tk.Label(self, text='Y =').grid(row=5,column=5)
			entryG01Y=tk.Entry(self, textvariable=G01Y, width=8).grid(row=5,column=6)
			
			
			
			
			#G02
			radioG=tk.Radiobutton(self,text='Arc en sentit horari (G02): ', value='G02', variable=ordre).grid(row=7,column=0)
			labelG02X=tk.Label(self, text='X =').grid(row=7,column=2)
			entryG02X=tk.Entry(self, textvariable=G02X, width=8).grid(row=7,column=3)
			labelG02Y=tk.Label(self, text='Y =').grid(row=7,column=5)
			entryG02Y=tk.Entry(self, textvariable=G02Y, width=8).grid(row=7,column=6)
			labelG02I=tk.Label(self, text='I =').grid(row=7,column=8)
			entryG02I=tk.Entry(self, textvariable=G02I, width=8).grid(row=7,column=9)
			labelG02J=tk.Label(self, text='J =').grid(row=7,column=11)
			entryG02J=tk.Entry(self, textvariable=G02J, width=8).grid(row=7,column=12)
			labelG02A=tk.Label(self, text='Orientacio =').grid(row=7,column=14)
			entryG02A=tk.Entry(self, textvariable=G02A, width=8).grid(row=7,column=15)
			
			
			
			
			
			
			#G03
			radioG=tk.Radiobutton(self,text='Arc en sentit horari (G03): ', value='G03', variable=ordre).grid(row=9,column=0)
			labelG03X=tk.Label(self, text='X =').grid(row=9,column=2)
			entryG03X=tk.Entry(self, textvariable=G03X, width=8).grid(row=9,column=3)
			labelG03Y=tk.Label(self, text='Y =').grid(row=9,column=5)
			entryG03Y=tk.Entry(self, textvariable=G03Y, width=8).grid(row=9,column=6)
			labelG03I=tk.Label(self, text='I =').grid(row=9,column=8)
			entryG03I=tk.Entry(self, textvariable=G03I, width=8).grid(row=9,column=9)
			labelG03J=tk.Label(self, text='J =').grid(row=9,column=11)
			entryG03J=tk.Entry(self, textvariable=G03J, width=8).grid(row=9,column=12)
			labelG03A=tk.Label(self, text='Orientacio =').grid(row=9,column=14)
			entryG03A=tk.Entry(self, textvariable=G03A, width=8).grid(row=9,column=15)
			
			
			
			
			
			#Rotacio
			radioG=tk.Radiobutton(self,text='Rotacio: ', value='apuntar', variable=ordre).grid(row=11,column=0)
			labelRA=tk.Label(self, text='Orientacio =').grid(row=11,column=2)
			entryRA=tk.Entry(self, textvariable=RA, width=8).grid(row=11,column=3)
			
			checkBox1=tk.Checkbutton(self, variable=checkBoxGraus, text="Orientacio en graus").grid(row=11, column=5)
			
			
			
			#Botons
			buttonDibuixar=tk.Button(self, text='Dibuixar', fg='blue', command=lambda:Dibuixar(self)).grid(row=20,column=7)
			buttonVeure=tk.Button(self, text='Veure', fg='blue', command=lambda: Veure(self)).grid(row=20,column=3)
			buttonUndo=tk.Button(self, text='Desfer', fg='blue', command=lambda: undo(self)).grid(row=20,column=5)
			buttonTorna=tk.Button(self, text='Tornar a l\'inici', fg='blue', command=lambda:back(self)).grid(row=20,column=9)
			
			canvas = tk.Canvas(self, width=500, height=500)
			canvas.grid(row=25,column=1,columnspan=12)
			
			turtle1 = turtle.RawTurtle(canvas)
			turtle1.shape("turtle")
			turtle1.setheading(90)
			turtle1.radians()
			
			label1=tk.Label(self, text='          ').grid(row=11,column=5)
			
			
			
			
			def Dibuixar(self):
				Draw.dibuixaLlista()
			
			
			def Veure(self):
				print ordre.get()
				tria=ordre.get()
				if tria == 'G00':
					x=G00X.get()
					y=G00Y.get()
					Draw.dibuixG00(turtle1,x,y)
				
				elif tria == 'G01':
					x=G01X.get()
					y=G01Y.get()
					Draw.dibuixG01(turtle1,x,y)
				
				elif tria == 'G02':
					x=G02X.get()
					y=G02Y.get()
					i=G02I.get()
					j=G02J.get()
					angle=G02A.get()
					graus=checkBox1.get()
					if graus == True:
						angle= (angle*math.pi)/180
					print (x, y, i , j, angle)
					Draw.dibuixG02(turtle1,x,y,i,j,angle)
				
				elif tria == 'G03':
					x=G03X.get()
					y=G03Y.get()
					i=G03I.get()
					j=G03J.get()
					angle=G03A.get()
					graus=checkBox1.get()
					if graus == True:
						angle= (angle*math.pi)/180 
					Draw.dibuixG03(turtle1,x,y,i,j,angle)
				
				elif tria == 'apuntar':
					angle=RA.get()
					graus=checkBox1.get()
					if graus == True:
						angle= (angle*math.pi)/180
					else:
						pass
					Draw.dibuixapuntar(turtle1,angle)
				
				else:
					pass
			
			def undo(self):
				Draw.desfer(turtle1)
			
			
			def back(self):
				turtle1.reset()
				turtle1.setheading(math.pi/2.0)
				controller.show_frame(StartPage)
	
	
	
	app = TFM()
	app.mainloop()
	
\end{python}

\section{Arduino}

\subsection{Robot.ino}\label{ArduinoProg}

\begin{lstlisting}[style=Arduino]
	#include <AccelStepper.h>
	#include <MultiStepper.h>
	#include <Servo.h>
	#include <SoftwareSerial.h>
	
	SoftwareSerial bluetooth(10,11);
	
	Servo boli;
	int up=0;
	int down=50;
	
	AccelStepper RodaDreta(1,9,8);
	AccelStepper RodaEsquerra(1,13,12);
	MultiStepper Robot;
	int VelocitatMax=500;
	long posicio[2];
	int pasdreta=0; //posicio absoluta del motor dret en pasos
	int pasesquerra=0; //posicio absoluta del motor dret en pasos
	
	int text[3]; 
	int cnt=0;
	boolean Rebut = false;
	
	
	
	
	
	void setup() {
		bluetooth.begin(9600);
		
		
		pinMode(2, OUTPUT); //microstepping off
		pinMode(3,OUTPUT);
		digitalWrite(2,HIGH);
		digitalWrite(3,HIGH);
		pinMode(4, OUTPUT);
		pinMode(5,OUTPUT);
		digitalWrite(4,HIGH);
		digitalWrite(5,HIGH);
		boli.attach(6);
		boli.write(0);
		
		Robot.addStepper(RodaDreta);
		Robot.addStepper(RodaEsquerra);
		
		RodaDreta.setMaxSpeed(VelocitatMax);
		RodaEsquerra.setMaxSpeed(VelocitatMax);
	}
	
	void loop() {
		getSerialData();
		delay(1);
		processData();
	}
	
	void getSerialData(){
		if(bluetooth.available() > 0) {
			String x = bluetooth.readString();
			String buff="";
			
			for (int i=0; i<x.length();i++){
				String caracter="";
				caracter=caracter+x[i];
				if (caracter!=","){
					buff=buff+x[i];
				}
				else {
					int y=buff.toInt();
					text[cnt]=y;
					buff="";
					if (cnt<3){
						cnt+=1;
						}
					if (cnt==3){
						cnt=0;
					}
				}
			}
			Rebut=true;
			delay(1);    
		}
	}
	
	
	void processData(){
		if (Rebut==true){
			if (text[0]==0 and boli.read()!=up){
				boli.write(up);
				delay(300);
			}
			else if (text[0]==1 and boli.read()!=down){
				boli.write(down);
				delay(300);
			}
			posicio[0]=-text[1];
			posicio[1]=text[2];
			Robot.moveTo(posicio);
			Robot.runSpeedToPosition();
			bluetooth.println("Ready");
			Rebut=false; 
		}
	}
\end{lstlisting}
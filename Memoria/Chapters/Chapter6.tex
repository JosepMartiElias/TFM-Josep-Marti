\setlength\topmargin{8mm}
\onehalfspacing
\chapter{Conclusions and future work} % Main chapter title

\label{Chapter6} % For referencing the chapter elsewhere, use \ref{Chapter1} 

\rhead[\emph{Study of the tritium production in a 1-D blanket model with Monte Carlo methods}]{\thepage}
\lhead[\thepage]{\emph{Study of the tritium production in a 1-D blanket model with Monte Carlo methods}}

\section{Conclusions}

The method to simulate a parallelepiped of rectangular section applying boundary conditions to its walls that act as this prism is stacked with other identical prisms has been tested and it seems to provide correct results for the specular boundary condition with the MCNP5 code. This method can be useful to collapse 3D geometries into monodimensional models.

The data of the mono dimensional collapse of the WCSB blanket from the IFERC have been used for the simulations and the neutron and photon flux profiles and energy deposition through the blanket have been obtained.

The TBR associated to this profile flux for this type of breeder have been calculated resulting in a generation of 1.4 tritium atoms for every neutron coming from the plasma. This value is reasonable but higher than the expected.

A comparision beetwen the default library of the MCNP5 and the JEFF library have been made, it can be observed higher values of neutron flux using the default library. 

This results are able to be integrated at the AINA code developed at the FEEL.

\section{Future work}

The results of this work should be verificated with other sources. A non estocastic but deterministic code like SCALE is a good candidate for this verification.

If it is possible, compare the results with other similar studies made by the investigators of the Japan Atomic Energy Agency.

A deeper study which explains the differences of the results beetwen the different libraries could be interesting. Also prepare the libraries that didn't work can be useful for future projects.

 
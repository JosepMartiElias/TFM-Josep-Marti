\setlength\topmargin{8mm}
\onehalfspacing
\chapter{Impacte medioambiental i social} % Main chapter title

\label{Chapter8} % For referencing the chapter elsewhere, use \ref{Chapter1} 

\rhead[\emph{Disseny, programació i implementació d'un robot de dibuix amb Arduino}]{\thepage}
\lhead[\thepage]{\emph{Disseny, programació i implementació d'un robot de dibuix amb Arduino}}



En aquest apartat s'estudiarà l'impacte medioambiental que ha tingut la realització d'aquest projecte. El primer punt a estudiar és el consum elèctric, ja que és imprescindible pel funcionament del mateix. El consum nominal dels motors és de 3 W i és, amb diferència, el consum més elevat. Per aquesta raó les possibles emissions de $CO_{2}$ que es produiran són gairabé negligibles. El principal element que pot tenir un impacte medioambiental negatiu és la bateria de liti, que és un element molt contaminant. És per això que se n'ha de fer un ús responsable, intentant mantenir-la sempre amb més del 50\% de la càrrega i apagant-la quan s'acaba d'utilitzar per evitar que es faci malbé. Un cop la bateria s'hagi de canviar, s'haurà de reciclar degudament portant-la a un centre de recollida de residus. 

Al llarg del procés hi ha hagut alguns procediments que poden haver tingut un impacte negatiu sobre el mediambient. En primer lloc, al realitzar les soldadures, ja que al realitzar-les s'emeten gasos nocius per la salut i per aquest motiu, es van realitzar amb la ventilació adient per evitar-ho. En quant a la impressió 3D i el seu material s'adjudica aquest responsabilitat a l'empresa subministradora obligada a cumplir amb la normativa. A part d'això, tots els elements que s'han consumit durant el procés de realització s'han reciclat com és degut, i tots els papers utilitzats per realitzar les proves han estat fulls reutilizats. 

Per altra banda, aquest robot podria tenir un impacte social molt positiu, ja que permetria realitzar feines de suport per alguns professionals de manera ràpida i a baix cost que ajudaria al seu desenvolupament. 

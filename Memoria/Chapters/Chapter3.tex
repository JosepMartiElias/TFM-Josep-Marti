% Chapter 3
\setlength\topmargin{8mm}
\onehalfspacing
\chapter{Disseny} % Main chapter title

\label{Chapter3} % For referencing the chapter elsewhere, use \ref{Chapter1} 

\rhead[\emph{Disseny, programació i implementació d'un robot de dibuix amb Arduino}]{\thepage}
\lhead[\thepage]{\emph{Disseny, programació i implementació d'un robot de dibuix amb Arduino}}

En aquest apartat es descriu el disseny del robot, la creació de l’estructura i la distribució dels components, explicant el perquè d’aquesta distribució i les permeses que s’ha tingut en compte alhora de dissenyar les diferents peces en Solid Works per tal d'imprimir-les posteriorment a l'aula Rep Rap de la universitat.   

\section{Xassís}

El xassís és el cos del robot, incorpora totes les altres peces i el seu disseny és una de les parts fonamentals del treball. És l’encarregat de centrar els eixos dels motors i fixar la seva posició, d’incorporar la bateria, el microcontrolador, els divers i tot el circuit elèctric i d'assegurar la seva posició fixe durant el moviment del robot. 
 
L’objectiu era dissenyar un xassís amb les dimensions més petites possibles, i aquestes van venir donades directament per la mida dels dos motors. S’ha dissenyat  amb una amplada de X tenint en compte que cada motor necessita com a base de suport un rectangle de X x X mm i a l’espai entre mig dels dos cal adaptar el suport del retolador. Per tal de no posar unes limitacions molt restrictives, s’ha decidit realitzar un forat per l’element de dibuix de XX mm de diàmetre per tal de poder adaptar el suport a les mides del retolador o estri escollit. És fonamental que aquest forat estigui centrat entre els dos motors i amb un eix perpendicular i coincident amb els eixos dels motors que han d’estar alineats. D’aquesta manera s’aconsegueix l’amplada mínima, ja que la resta d’elements són de dimensions més reduïdes i per tant s’han posicionat de manera centrada per mantenir centrat el centre de masses. 

Per tal de fixar els motors s’han incorporat uns suports verticals que fixen amb cargols la cara frontal del motor (on neix l’eix) evitant així que pugui desplaçar-se i mantenint d’aquesta manera els dos motors sempre alineats. Aquest suports es van imprimir incorporats directament al xassís, per minimitzar els possibles errors de muntatge o descentratge. De la mateixa manera, aprofitant aquests suports i per optimitzar l’espai, s’ha incorporat una petita part plana horitzontal sobre els motors per instal·lar-hi els drivers dels mateixos i així mantenir-ho junt i facilitant el reconeixement del driver corresponent a cada motor. Cal destacar que aquest suports no tanquen la part superior de tal manera que es muntar posar i desmuntar els motors de manera més còmode sense, per exemple, desmuntar el suport del retolador. 

La longitud del xassís ha de ser la suficient per col·locar-hi els motors, el servomotor encarregat d’aixecar el retolador, la bateria, el microcontrolador Arduino i la roda davantera. D’aquesta manera, el xassís final te una llargada de XXX mm i la següent distribució: 

En primer lloc es posicionen els motors en un espai de XX mm  (fixació?). Seguidament hi ha un espai reservat al suport fixe del bolígraf que s’ha dissenyat com una peça diferent per tal de poder adaptar-la a altres utillatges sense haver de modificar el xassís per complet. Disposa d’una base per enganxar el servomotor fixant la seva alçada de manera que l’eix del servo es mantingui a la mateixa alçada que l’extrem del suport fixe, per tal d’exercir la força per aixecar el retolador sobre el suport mòbil des de la posició horitzontal per gaudir del recorregut màxim del moviment si fos necessari. 

En següent lloc s’ha dissenyat un espai reservat per la bateria el qual compte amb un petit compartiment que manté la bateria en posició vertical i fixe per evitar que caigui o pugui interrompre el funcionament del robot. Aquest compartiment té una mida de XxX amb tres parets de Xmm d’alçada i la paret posterior de Xmm que s’aprofita per fer de base de l’Arduino. Es pot observar que les mides són uns mil·límetres més grans que la pròpia bateria ja que s’ha deixat l’espai suficient per tal de poder posar els cargols per fixar l’Arduino, és per aquesta raó que s’ha deixat un petit espai entre les parets laterals i aquesta per poder manipular els cargols amb la bateria fixada si fos necessari. S’ha decidit instal·lar la bateria al punt mig del robot per distribuir la massa al llarg del xassís i evitant que el centre de masses quedi a la línia de l’utillatge de dibuix per evitar que el robot pugui bolcar en algun punt de la trajectòria degut a algun moviment concret.   

L’Arduino, com s’acaba d’explicar es col·loca a la paret vertical del compartiment de la bateria de manera que queda fixat verticalment, optimitzant així els espais. S’han dissenyat els forats per tal de col·locar l’entrada del port serial a la part superior fent possible d’aquesta manera la programació de la placa un cop s’ha implementat tot el prototip, ja que sinó s'hauria de desmontar cada cop que es volgues programar. Al lateral de la placa s’han instal·lat dues barres de corrent de protoboard per tal de poder alimentar des de la mateixa sortida de la font un motor conjuntament amb la placa i l’altre motor amb el servomotor. 

Per acabar, a la part davantera s’ha deixat l’espai lliure suficient per instal·lar-hi una roda boja que actuarà com a punt de suport per estabilitzar el robot. 

\section{Guia del retolador}

Aquest suport esta dissenyat per funcionar com a guia pel retolador, de manera que el seu desplaçament sigui vertical i centrat per minimitzar l’error de descentratge del mateix. Consta d’una part plana en forma de T la qual anirà fixada al xassís amb 3 cargols i un tub vertical que farà de guia pel retolador. Aquest tub està expressament dissenyat per treballar amb un retolador Edding 1200 (presentat a l’apartat (\ref{sec:retolador})) i s’ha dissenyat de manera independent del xassís per així poder fer-la a mida de l’estri que es vulgui utilitzar sense haver de tornar a imprimir tot el cos del robot. 

A part de la guia per l’estri de dibuix, també compta amb un petit suport per tal d’enganxar el servomotor de manera que l’eix d’aquest coincideixi amb l’alçada de la guia i aconseguir que el moviment del servo pugui ser màxim, des de la posició horitzontal amb l’estri actiu fins a una posició mitja amb l’estri aixecat i per tant, sense actuar. No és més que un petit graó a la part dreta del suport de les mides del servo i l’alçada de Xmm per aconseguir la posició desitjada del servo, que s’ha fixat directament amb cinta adhesiva de doble cara ja que el seu esforç no és massa elevat i queda el fixe perfectament sense haver de foradar el suport imprès en 3D. 

\section{Mecanisme d'elevació del retolador} \label{sec:suportmobil}

Consta d’una petita placa fixada a certa altura del retolador que actua com a superfície per tal de poder transmetre el moviment circular del servo en el moviment vertical de l’utillatge. L’alçada a la qual s’ha de col·locar ve donada per la guia del moviment fixe al xassís, assegurant aquesta manera que quan el servomotor està en posició de dibuix aquest suport es recolzi a la cara plana superior de la guia i el retolador estigui en contacte amb el paper. 

S’ha dissenyat amb una part més allargada per tal d’assegurar el contacte i l’acció del servomotor al aixecar el braç, i al altre costat s’ha deixat un espai buit i dues parets verticals foradades per tal de fixar l’estri amb l’ajuda d’un cargol un cop ajustada l'altura a la qual s'ha de fixar, mantenint el servo en posició activa, el retolador tocant al paper i el suport recolzat a la cara superior de la guia. L’eix vertical serveix per tal d’assegurar la perpendicularitat de l’estri amb la base del suport. 

\section{Roda boja davantera}

Com a tercer punt de suport i per estabilitzar el robot, s’ha dissenyat una roda boja de manera que no intervingui en el moviment del robot, permetent el lliscament en qualsevol direcció. Consta de dues parts, el cos imprès en 3D i una bola de vidre que gira lliurament a l'interior. El cos assegura el contacte amb la bola en només 4 punts d’una circumferència paral·lela a la base fent així que el lliscament sigui més fàcil i minimitzant la força de fricció. També assegura que la bola es mantingui dins de la cavitat quan s’aixeca el robot fet que provoca que la bola entri a pressió. La cara superior és una cara plana per facilitar la impressió 3D de la peça amb dos forats per fixar-la amb cargols al xassís i un forat per on es pot observar la bola per poder treure-la aplicant una petita pressió. 

L’alçada de la roda, junt amb les rodes motrius, són les encarregades fixar l’alçada del robot, de manera que s’han dissenyat conjuntament per tal d’assegurar que la base sigui paral·lela al paper i mantenir l’eix del retolador vertical i centrat. 

\section{Rodes motrius}

Les rodes són una part fonamental del disseny, ja que són les encarregades del moviment del propi robot. Hi ha quatre aspectes bàsics per a la seva construcció: minimitzar la superfície de contacte amb el paper, assegurar la perpendicularitat i concentricitat amb l’eix, evitar el lliscament perquè giri de manera solidaria al eix i optimitzar el diàmetre per aconseguir un pas més precís disminuint al màxim el diàmetre. 

Per minimitzar la superfície de contacte amb el paper, s’ha decidit utilitzar una junta tòrica de goma d’1,5mm de diàmetre, que actua de la mateixa manera que els neumàtics d’un cotxe, evitant el lliscament amb el paper i, al ser circular, es minimitza el contacte fent-lo puntual o gairebé puntual. Així s’evita que un major contacte de la roda pugui actuar en contra del moviment del robot creant forces de fricció que evitin la rotació del robot sobre el punt de contacte. Amb una superfície puntual també es millora la precisió dels moviments circulars, ja que es calculen a partir de la distància entre el retolador i el punt de contacte, i per tant, com més petit sigui aquest últim, més exacte i constant serà aquesta distància durant el moviment radial de la roda. Per tal de mantenir el neumàtic a la roda, s’ha imprès directament amb un petit conducte amb la forma de mig neumàtic aconseguint així que encaixin. 

En la pròpia impressió de la roda, s’ha afegit un eix transversal del mateix diàmetre que l’eix del motor per assegurar-ne d’aquesta manera la perpendicularitat i concentricitat. És un fet clau per assegurar el correcte funcionament, perquè, d’igual manera que al punt anterior, s’ha d’assegurar que la distancia entre les rodes sigui sempre la mateixa al llarg de tots els punts de la rotació de les mateixes, ja que és així com s’ha calculat la trajectòria. 

Un altre element del disseny són les dues petites parets verticals foradades que surten de l’eix i el forat que es pot observar al cos de la roda. Això permet que amb un cargol i una femella es pugui fixar la roda al eix ja que es crea una deformació ínfima que redueix el forat per on passa l’eix contra el propi eix fent que quedin fixats sense alterar la forma de la roda. El forat a la pròpia roda és d’aquestes dimensions per tal de permetre la manipulació dels cargols del motor sense la necessitat de treure la roda. 

Per acabar, el diàmetre de la roda s’ha intentat minimitzar de manera que es disminueix l’avanç de la roda per cada pas, augmentant així la precisió. Això s’ha realitzat tenint en compte que la distancia entre l’eix del motors i la base més una distància prudencial de XXXmm per poder col·locar els cargols necessaris per fixar els suports de l’utillatge de dibuix i la roda boja. D’aquesta manera s’han dissenyat unes rodes de XXXmm que sumat als neumàtics de goma

\section{Cargols, femelles i tubs auxiliars}

Per l’assemblatge de tot el prototip s’han utilitzat cargols de mètrica M3 i les femelles adients per tal de fixar-los. S’han utilitzat un total de 14 cargols M3x10mm de llargada, 3 per fixar el suport del retolador, 3 per fixar la placa Arduino al xassís i 8 per fixar els motors als suports del xassís, 4 per cada motor. Per altre banda, s’han emprat 4 cargols M3x30mm per fixar els divers (2 per cadascun) i 2 més per la roda boja davantera, que es podia fixar amb cargols més curts, però s’ha decidit per aquests ja que estaven en disposició. Tots els cargols porten una femella per tal de poder fer fixe la unió, i, en el cas dels motors, perquè es necessitava separar els cargols dels cargols de tancament posteriors del motor amb els quals comparteixen eix, i amb una femella s’ha escurçat la llargada de 10mm ja que eren els cargols dels quals es disposava. 

Per altre banda, s'han dissenyat i imprés en 3D uns tubs de plàstic, per tal d’elevar tant els divers com la placa Arduino. En primer lloc s’han aixecat els divers XXXmm respecte als seus suports per permetre d’aquesta manera la correcta connexió dels cables que necessita la qual es fa per la part inferior del mateix. Per altra banda, el microcontrolador s’ha separat XXXmm respecte la paret per així evitar el contacte dels pins soldats i evitar possibles sobreescalfament. 


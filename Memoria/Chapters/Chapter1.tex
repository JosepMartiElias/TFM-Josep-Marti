% Chapter 1
\fancyfoot[LE,RO]{
	\includegraphics[scale=0.3]{logo1}
     }
\setlength\topmargin{8mm}
\onehalfspacing
\chapter{Introducció} % Main chapter title

\label{Chapter1} % For referencing the chapter elsewhere, use \ref{Chapter1} 

\rhead[\emph{Disseny, programació i implementació d'un robot de dibuix amb Arduino}]{\thepage}
\lhead[\thepage]{\emph{Disseny, programació i implementació d'un robot de dibuix amb Arduino}}

%----------------------------------------------------------------------------------------

Imaginem un arquitecte que pretén delinear sobre un terreny per tal d'indicar els espais que ocuparà la construcció. Aquest necessitarà contractar un topògraf per tal de realitzar-ho, fet que provoca un augment de pressupost i dels temps d'execució de l'obra. Suposem ara que es vol realitzar una instalació elèctrica en un carrer i, a l'hora de foradar el paviment, existeix el perill de fer malbé les canonades que hi passen per sota. Actualment, es disposa del plànol de canonades però és difícil determinar amb exactitud per on passen. Per altra banda, imaginem que es vol tallar sobre una planxa de fusta, de cartró, un taulell DM o algún altre material, un disseny específic. Per fer-ho primer cal traspassar el disseny a la superficie per tal de poder-ho tallar després. Per acabar, suposem que hi ha la necessitat de traçar les línies de qualsevol espai esportiu, com un camp de futbol o una pista d'atletisme, i es vol evitar que puguin quedar les línies desviades. Aquests són alguns exemples de situacions reals en les que es requereix un trasspas d'un disseny tècnic o plànol a una superficie plana. 

Aquest projecte vol estudiar la viabilitat de realitzar un dispositiu de baix cost, fàcilment transportable i que es pugui utilitzar per resoldre qualsevol d'aquestes situacions. Aixó donarà peu a una eina de dibuix que eliminarà les possibles limitacións de tamany que presenten les actuals. S'estudiarà una primera aproximació a la solució, realitzant un prototip capaç de convertir arxius SVG (Scalar Vector Graphics) que contiguin dibuixos tècnics, al movient d'un robot capaç de dibuixar sobre una superficie plana sense necessitat d'utilitzar grans estructures. 

S'intentarà realitzar a partir de la teconologia de la qual es disposa: en primer lloc cal estudiar la manera de convertir un disseny tècnic a una trajectòria. A continuació, s'ha de programar el robot per aconseguir que segueixi aquesta trajectòria. Per fer-ho, primer cal escollir quins components poden aconseguir aquest moviment de la manera més precisa possible i estudiar el seu funcionament, així com definir un mètode de control que els pugui fer funcionar conjuntament. Per altra banda, és necessari realitzar el disseny de l'estructura que incorpori tots aquests elements per tal de crear un robot móbil de petites dimensions. Un cop realitzat el disseny i construit el robot, s'ha d'estudiar la precisió del mateix per veure si compleix amb les especificacions.

\section{Objectius}

Aquest projecte pretén dissenyar, programar i construir un robot capaç de dibuixar sobre un paper, de qualsevol dimensió, la figura o traçada que l'usuari desitgi. Per tal d'aconseguir-ho, s'han un definit uns objectius principals:

\begin{itemize}
	\item Avaluar la viabilitat de realitzar el robot com un sistema de llaç obert.
	\item Dissenyar i construir un robot des de zero.
	\item Dotar el robot dels mitjants de software necessaris per a la realització de la seva funció. 	
\end{itemize}

Per poder complir els objectius principals s'han d'assolir primer alguns objectius secundaris. 

\begin{itemize}
	\item Extreure el GCode associat a la trajectòria de la figura que l'usuari vol dibuixar.
	
	\item Crear l'algoritme encarregat de traduir un arxiu de text que conté el GCode associat a una trajectòria, a les ordres necessàries per moure el robot.
	
	\item Seleccionar els elements que millor s'ajustin a les especificacions del projecte, per tal de garatir el funcionament i augmentar la precisió.
	
	\item Aprofundir en la programació tant d'Arduino com de Python, per poder realitzar el projecte, i en el disseny CAD amb Solidworks per tal de dissenyar les peces que s'imprimiran en 3D. Per altra banda, cal aprendre a utilitzar altres programes o llenguatges, com l'Inkscape, per a la realització de dibuixos, i el Latex per a la realització de la memòria.
	
	\item Establir un protocol de comunicació entre el robot i l'ordinador.
	
	\item Programar una aplicació que permeti a l'usuari treballar amb el robot de manera més senzilla i entenedora.
\end{itemize} 



\section{Abast i resultat esperat}

L'abast d'aquest projecte és dissenyar, programar i construir un prototip del robot que funcioni en llaç obert, capaç de seguir les trajectòries desitjades per l'usuari, i programar l'aplicació per utilitzar-lo. A partir d'aquí, cal aclarir alguns punts que queden fora d'aquest abast per motius diferents:

\begin{itemize}
	\item En primer lloc, cal deixar clar que es vol estudiar la viabilitat de realitzar els objectius amb un sistema en llaç obert i, per tant, queda fora de l'abast del projecte l'estudi d'implementació d'un sistema que permeti tancar el llaç de control. Aixó s'ha decidit així per centrar l'estudi en la part inicial de la construcció i programació del robot, ja que el volum de feina que pot comportar tancar el llaç de control podria constituir un altre treball per si mateix, i per tant s'ha descartat. 
	
	\item Aquest projecte vol trobar una primera solució i estudiar-ne la seva viabilitat, per tant queda fora de l'abast aconseguir millorar les seves prestacions, ja que aixó voldira dir realitzar grans inversions en millores de materials i components, i una feina de redisseny per aconseguir-ho.
	
	\item S'ha descarat la creació d'un mètode d'extracció del GCode associat a una imatge, ja que ja existeixen eines que ho realitzen, com el programa Inkscape. Gràcies a l'existència d'aquests s'ha decidit no crear un nou software per aconseguir-ho i així destinar aquest temps a altres tasques més importants.
	
	\item Al ser un treball d'estudi i en el qual es vol crear un primer prototip per estudiar-ne la viabilitat s'ha descartat l'opció de crear una placa de circuit imprés (PCB) que integri tots els components que el formen, ja que és un procés de millora continua i s'han d'afegir nous components al llarg del temps.   
\end{itemize}











